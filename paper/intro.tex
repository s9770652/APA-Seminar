\section{Introduction}
\label{sec:intro}

\begin{itemize}
	\item
	problem introduction, motivation, applications

	\item
	formal problem definition (incl. why geometric mean?)

	\item
	short literature review: What is known, what not? New findings?

	\item
	content \& structure of paper
\end{itemize}

\begin{definition}
	Let~\(\goods\) be a set of size \(m\) and~\(\agents\) be a set of size \(n\).
	The elements of set~\(\goods\) are called \emph{items}, and the elements of set~\(\agents\) are called \emph{agents}.
	An \emph{allocation} is a tuple \(\alloc[][] = (\alloc)_{i \in \agents}\) of so-called \emph{bundles} \(\alloc \subset \goods\) such that each item is element of exactly one bundle, that is \(\mathop{\bigcup\hspace{-1pt}_{i \in \agents}} \alloc = \goods\) and \(\alloc \cap \alloc[][i'] = \emptyset\) for all \(i \neq i'\).
	An item~\(j \in \goods\) is \emph{assigned}\todo{or rather \enquote{allocated}?} to agent \(i \in \agents\) if \(j \in \alloc\) holds.
\end{definition}
\todo[inline]{mention divisibility}

\(\vdots\)

\begin{problem}
	Given a set \(\goods\) of items and a set \(\agents\) of agents with \emph{valuation functions} \(\valuations \colon \powerset[\goods] \to \real\) and \emph{agent weights} \(\eta_i\) for all agents \(i \in \agents\), the \emph{Nash Social Welfare problem} (\NSW) is to find an allocation maximising the weighted geometric mean of valuations, that is
	\begin{equation*}
		\argmax_{\alloc[][] \in \allallocs{n}{\goods}} \braces[\bigg]{ \paren[\Big]{ \smashoperator{\prod_{i \in \agents}} \valuations[\alloc]^{\weight} }^{1 / \sum_{i \in \agents} \weight} }
	\end{equation*}
	where \(\allallocs{n}{\goods}\) is the set of all possible allocations of the items in \(\goods\) amongst \(n\) agents.
	The problem is called \emph{symmetric} if all agent weights \(\weight\) are equal, and \emph{asymmetric} otherwise.
\end{problem}
\todo[inline]{agents without items assigned have valuation zero → prevent}

\(\vdots\)

In a slight abuse of notation, we omit curly braces delimiting a set in the arguments of a valuation function, so we write \(\valuations[j_1, j_2, \dots]\) but mean \(\valuations[\{j_1, j_2, \dots\}]\) for example.

\(\vdots\)

\todo[inline]{definition of approximation factor [def environment or in-text?]}

\(\vdots\)

\Citeauthor{APNSWuSVþUM}~\cite{APNSWuSVþUM} consider five different types of non-negative monotonically non-decreasing valuation functions of which we are going to consider only the following two due to space constraints:
\begin{description}
	\item[Additive]
	The valuation \(\valuations[\genericset]\) of an agent \(i\) for a set \(\genericset \subset \goods\) of items \(j\) is the sum of individual valuations \(\valuations[j]\), that is \(\valuations[\genericset] = \sum_{j \in \genericset} \valuations[j]\).

	\item[Submodular]
	Let \(\valuations[\genericset[1] \given \genericset[2]] \coloneq \valuations[\genericset[1] \cup \genericset[2]] - \valuations[\genericset[2]]\) denote the marginal valuation\todo{or rather utility?} of agent~\(i\) for a set \(\genericset[1] \subset \goods\) of items over a \emph{disjoint} set \(\genericset[2] \subset \goods\).
	This valuation functions satisfies the submodularity constraint \(\valuations[j \given \genericset[1] \cup \genericset[2]] \le \valuations[j \given \genericset[1]]\) for all agents \(i \in \agents\), items \(j \in \goods\) and sets \(\genericset[1], \genericset[2] \subset \goods\) of items.
	\todo[inline]{What is the motivation for submodular functions? [IRfCAwSUF]}
\end{description}
We use \emph{additive \NSW} and \emph{submodular \NSW} as shorthands for the Nash social welfare problems with additive and submodular valuation functions, respectively.

\begin{itemize}
	\item
	non-sharable indivisible resources -> items; opposed to divisible and sharable ones like money, electricity, network connections

	\item
	different types of efficiency: Pareto optimal, utilitarian, egalitarian, Nash

	\item
	additive:
	\NP-hard 1,069 (https://www.semanticscholar.org/reader/1c5347490b006dc6c498131a9ee01281a31edbe7)

	\item
	submodularity:
	discrete analogue of concavity and arises naturally in economic setting since it caputres the property that marginal utilities are decreasing as we allocate more goods
	\begin{itemize}
		\item
		alt def: https://www.semanticscholar.org/reader/1c5347490b006dc6c498131a9ee01281a31edbe7
	\end{itemize}
	matroids?

	\item
	NSW [Lee17] [CKM\textsuperscript{+}16]:
	\NP-hard 1,5819, EF1, PO, approximation recover most guarantees;
	https://www.semanticscholar.org/reader/1c5347490b006dc6c498131a9ee01281a31edbe7:
	invariance under scaling

	\item
	mention Nash's name in footnote

	\item
	especially asymmetric case less understood
	https://www.semanticscholar.org/reader/bef4514574543720bc45ea235df4c4556e97efe0

	\item
	Utilitarian SW completely resolved [Von08] [KLMM08]:
	constant-factor \(\frac{\euler}{\euler - 1}\), hardness result

	\item
	maximin somewhat resolved [AS10] [KP07] [BD05]:
	additive \(\O(\sqrt{n} \log^3 n)\)-factor;
	submodular \(\O(n)\)-factor;
	hardness of \(2\) for both

	\item
	symmetric additive(-like) NSW [BKV18]:
	\(\euler^{1/\euler}\)-factor

	\item
	asymmetric submodular NSW [NR14] [CDG\textsuperscript{+}17]:
	not much known;
	\(\O(m)\)-factor

	\item
	applications
	\begin{itemize}
		\item
		industrial procurement:
		capture of preferences?
		in bundles or singular?
		capture of business rules (physical objects, configuration, synergy)
		\cite{survey}

		\item
		earth observation satellite:
		co-founded -> weights reflecting investment -> utility must reflect that;
		sharable as data can be distributed multiple times
		\cite{survey}

		\item
		manufacturing system:
		task scheduling across production resources -> network of dependencies;
		dynamic, real-time calculation (feasibility) more important than optimality because of disturbance -> distributedway by means of cooperation and coordination; rescheduling
		\cite{survey}

		\item
		combinatorial auctions:
		FCC auction of spectrum licenses
		\cite{inapproximability}

		\item
		edge computing:
		\enquote{paradigm in which the resources for communication, computation, control and storage are placed at the edge of the Internet, in close proximity to mobile devices, sensors, actuators, connected things and end users.};
		highly competitive environment;
		locality, latency, security, incentivisation
		\cite{edge_computing_auction, edge_computing_report}
	\end{itemize}

	\item
	B. R. Chaudhury, Y. K. Cheung, J. Garg, N. Garg, M. Hoefer, and K. Mehlhorn. Fair division
	of indivisible goods for a class of concave valuations. J. Artif. Intell. Res., 74:111–142, 2022
\end{itemize}