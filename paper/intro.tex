\section{Introduction}
\label{sec:intro}

\begin{itemize}
	\item
	problem introduction, motivation, applications

	\item
	formal problem definition

	\item
	short literature review: What is known, what not? New findings?

	\item
	content \& structure of paper
\end{itemize}

\begin{definition}
	Let \(\goods \coloneq \{1, \dots, m\}\) be a set of indivisible \emph{items} and \(\agents \coloneq \{1, \dots, n\}\) be a set of \emph{agents}.
	An \emph{allocation} is a tuple \(\alloc[][] = ( \alloc[][1], \dots, \alloc[][n] ) \in \powerset[G]^n\) such that each item is element of exactly one set \(\alloc\), that is \(\mathop{\bigcup\hspace{-1pt}_{i \in \agents}} \alloc = \goods\) and \(\alloc \cap \alloc[][i'] = \emptyset\) for all \(i \neq i'\).
	An item \(j \in \goods\) is \emph{assigned}\todo{or rather \enquote{allocated}?} to agent \(i \in \agents\) if \(j \in \alloc\) holds.
\end{definition}

\(\vdots\)

\begin{definition}
	Given a set \(\goods\) of items and a set \(\agents\) of agents with \emph{valuations} \(\valuations \colon \powerset[\goods] \to \real\)\todo[info]{less strict def of valuations; restriction for our case later on} and \emph{agent weights} \(\eta_i\) for all agents \(i \in \agents\), the \emph{Nash Social Welfare problem} (\NSW) is to find an allocation maximising the weighted geometric mean of valuations, that is
	\begin{equation*}
		\argmax_{\alloc[][] \in \allallocs{n}{\goods}} \braces[\bigg]{ \paren[\Big]{ \prod_{i \in \agents} \valuations[\alloc]^{\weight} }^{1 / \sum_{i \in \agents} \weight} }
	\end{equation*}
	where \(\allallocs{n}{\goods}\) is the set of all possible allocations of the items in \(\goods\) amongst \(n\) agents.
	The problem is called \emph{symmetric} if all agent weights \(\weight\) are equal, and \emph{asymmetric} otherwise.
\end{definition}
\todo[inline]{agents without items assigned have valuation zero → prevent}

\(\vdots\)

In a slight abuse of notation, we omit curly braces delimiting a set in the arguments of a valuation function, so for example we write \(\valuations[j_1, j_2, \dots][]\) to denote \(\valuations[\{j_1, j_2, \dots\}][]\).

\(\vdots\)

\todo[inline]{definition of approximation factor [def environment or in-text?]}

\(\vdots\)

Garg, Kulkarni and Kulkarni~\cite{APNSWuSVþUM} consider five different types of non-negative monotonically non-decreasing valuation functions of which we are going to consider only the following two due to space constraints:
\begin{description}
	\item[Additive]
	The valuation \(\valuations[\genericset]\) of an agent \(i\) for a set \(\genericset \subset \goods\) of items \(j\) is the sum of individual valuations \(\valuations[j]\), that is \(\valuations[\genericset] = \sum_{j \in \genericset} \valuations[j]\).

	\item[Submodular]
	Let \(\valuations[\genericset[1] \given \genericset[2]] \coloneq \valuations[\genericset[1] \cup \genericset[2]] - \valuations[\genericset[2]]\) denote the marginal utility of agent~\(i\) for a set \(\genericset[1] \subset \goods\) of items over the disjoint set \(\genericset[2] \subset \goods\).
	This valuation functions satisfies the submodularity constraint \(\valuations[j \given \genericset[1] \cup \genericset[2]] \leq \valuations[j \given \genericset[1]]\) for all agents \(i \in \agents\), items \(j \in \goods\) and sets \(\genericset[1], \genericset[2] \subset \goods\) of items.
\end{description}
We use \emph{additive \NSW} and \emph{submodular \NSW} as shorthands for the Nash social welfare problems with additive and submodular valuation functions, respectively.

\lipsum[9-13]