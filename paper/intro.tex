\section{Introduction}
\label{sec:intro}

\subsection{Motivation}
\label{subsec:intro:motivation}

The study of distributing goods amongst several recipients is an interdisciplinary field, pursued scientifically as early as the 1940s~\cite{the_problem_of_fair_division}.
It is interesting both computationally (how to distribute fast) and qualitatively (how to distribute well).
Its areas of application are manifold:
\begin{itemize}
	\item
	In industrial procurement, the preferences of buyers and sellers need be appropriately captured and real-world constraints on goods and services be taken into account~\cite{survey}.

	\item
	Mobile Edge Computing denotes a technique where mobile devices compete for computing and storage capabilities of physically close servers.
	However, the participation of users and the provision of the serves has to be incentivised and monetised~\cite{edge_computing_auction, edge_computing_report}.

	\item
	Many production sites, machines, suppliers, etc. are required for manufacturing, and the tasks between them must be scheduled efficiently.
	Disturbances must be quickly paid heed to~\cite{survey}.

	\item
	Water is a crucial resource, and even hostile countries must come to mutual agreements on the withdrawal from contested rivers~\cite{water_management}.
\end{itemize}


\subsection{Preliminaries}
\label{subsec:intro:prelim}

In this seminar paper, we focus on unsharable and indivisible goods, which we term \emph{items}.
The recipients of those items are termed \emph{agents}.
The distributions of items amongst agents are modelled through allocations.
\begin{definition}
	Let~\(\goods\) be a set of \(m\) items and~\(\agents\) be a set of \(n\) agents.
	An \emph{allocation} is a tuple \(\alloc[][] = (\alloc)_{i \in \agents}\) of \emph{bundles} \(\alloc \subset \goods\) such that each item is element of precisely one bundle, that is, \(\bigcup_{i \in \agents} \alloc = \goods\) and \(\alloc \cap \alloc[][i'] = \emptyset\) for all \(i \neq i'\).
	An item~\(j \in \goods\) is \emph{assigned} to agent \(i \in \agents\) if \(j \in \alloc\) holds.
\end{definition}

The satisfaction of an agent \(i\) with her bundle \(\alloc\) is measured by her \emph{valuation function}~\(\valuations\), which assigns each item set a real value.
We always assume that valuation functions are monotonically non-decreasing, \ie, \(\valuations[\genericset[1]] \le \valuations[\genericset[2]]\), \(\forall \genericset[1] \subset \genericset[2] \subset \goods\), and normalised, \ie, \(\valuations[\emptyset] = 0\).
Note that this implies non-negativity, \ie, \(\valuations[\genericset] \ge 0\), \(\forall \genericset \subset \goods\).
Besides fulfilling these properties, the valuation functions can come from a plethora of function families.
We discuss additive and submodular functions in greater detail.
\begin{description}
	\item[Additive]
	The valuation \(\valuations[\genericset]\) of an agent \(i\) for any item set \(\genericset \subset \goods\) is the sum of the valuations of the individual items \(j \in \genericset\), that is, \(\valuations[\genericset] = \sum_{j \in \genericset} \valuations[\{j\}]\).

	Additive functions are fairly simple but also useful, and many expansions exist~\cite{satiation_in_fisher_markets_and_approx_of_nsw, APNSWuSVþUM}.

	\item[Submodular]
	Let \(\valuations[\genericset[1] \given \genericset[2]] \coloneq \valuations[\genericset[1] \cup \genericset[2]] - \valuations[\genericset[2]]\) denote the marginal valuation of agent~\(i\) for an item set \(\genericset[1] \subset \goods\) over a disjoint set \(\genericset[2] \subset \goods\).
	This valuation function satisfies the submodularity constraint \(\valuations[\{j\} \given \genericset[1] \cup \genericset[2]] \le \valuations[\{j\} \given \genericset[1]]\) for all \(\{j\}, \genericset[1], \genericset[2] \subset \goods\).

	Submodular valuation functions, which encompass the additive ones, have the property that the gain from assigning new items decreases with increasing bundle size.
	Diminishing returns are a common phenomenon in economics, making submodular functions worthwhile to study~\cite{inapprox_results_for_combi_auctions_with_submod_utility_funcs}.
	Their relations to matroids~\cite{submodular_low_value, approximating_nsw_under_rado_valuations, opt_approx_for_the_submod_nsw_in_the_value_oracle_model} make them interesting from a theoretical point of view, too.
\end{description}
In a slight abuse of notation, we sometimes omit curly braces delimiting a set, so we write \(\valuations[j_1, j_2, \dots]\) instead of \(\valuations[\{j_1, j_2, \dots\}]\) for example.

In order to measure and maximise the overall satisfaction of all agents, one needs to combine their valuations.
Several options arise here;
common choices are the utilitarian social welfare, that is the sum of all valuations~\cite{inapprox_results_for_combi_auctions_with_submod_utility_funcs, survey, APNSWuSVþUM, water_management, edge_computing_auction}, and the egalitarian social welfare, that is the minimum of all valuations~\cite{survey, APNSWuSVþUM}.
We consider a third one, the Nash social welfare.
\begin{problem}
	\label{prob:nsw}
	Given a set \(\goods\) of items and a set \(\agents\) of agents with valuation functions \(\valuations \colon \powerset{\goods} \to \realposzero\) and weights \(\eta_i \in \realpos\) for all agents \(i \in \agents\), the \emph{maximum Nash social welfare problem (\NSW)} is to find an allocation \(\alloc*[][]\) which maximises the weighted geometric mean of valuations, that is,
	\begin{equation*}
		\alloc*[][] \overset{!}{=} \smashoperator{\argmax_{\alloc[][] \in \allallocs{\scriptstyle\agents\kern1pt}{\goods}}} \braces{ \NSW(\alloc[][]) }
		\quad\text{with~}
		\NSW(\alloc[][]) \coloneq \paren[\Big]{ \smashoperator{\prod_{i \in \agents}} \valuations[\alloc]^{\,\textstyle\weight} }{}^{\textstyle 1 / \sum_{i \in \agents} \weight}
	\end{equation*}
	where \(\allallocs{\agents}{\goods}\) is the set of all possible allocations.
	The problem is called \emph{symmetric} if all agent weights \(\weight\) are equal, and \emph{asymmetric} otherwise.
\end{problem}

For the techniques employed in later sections, it is beneficial to consider the logarithmic Nash social welfare, that is,
\vspace{-1.5ex}
\begin{equation}
	\log \NSW(\alloc[][])
	= \frac{1}{\sum_{i \in \agents} \weight} \cdot \sum_{i \in \agents} \weight \log \valuations[\alloc] \mperiod[,]
\end{equation}
which is a sum instead of a product.
The Nash social welfare strikes a middle ground between the utilitarian and egalitarian social welfare, which focus on efficiency (height of overall satisfaction) and fairness (how agents value other agents' bundles), respectively.
In addition, it exhibits scale-freeness, that is invariance to the scales in which the valuations are expressed.
Even though the \NSW{} is \APX-hard, approximate solutions largely keep the properties of optimal allocations~\cite[see also \cref{rem:ef1}]{approximating_the_nsw_with_indiv_items, the_unreasonable_fairness_of_max_nsw, min_envy_and_max_avg_nsw_in_the_alloc_of_indiv_goods, finding_fair_and_efficient_allocs}.


\subsection[Related Work and Contribution]{Related Work and Contribution by \citeauthor{APNSWuSVþUM}}
\label{subsec:intro:contribution}

The research on the \NSW{} is rather young and less advanced than the research on other allocation problems.
As reference\footnotemark[1], for the submodular utilitarian social welfare problem, a hardness of \(\af\) was proven in \citeyear{inapprox_results_for_combi_auctions_with_submod_utility_funcs}~\cite{inapprox_results_for_combi_auctions_with_submod_utility_funcs}, and an \(\af\)-approximation algorithm was shown in \citeyear{opt_approx_for_the_submod_nsw_in_the_value_oracle_model}~\cite{opt_approx_for_the_submod_nsw_in_the_value_oracle_model} \Dash the additive case is trivially solvable through repeated maximum matchings anyway.
For the egalitarian social welfare problem, a randomised \((320 \sqrt{n} \log^3 n)\)-approximation algorithm for additive valuations~\cite{an_approx_algo_for_maxmin_fair_alloc_of_indiv_goods} and a \((2n-1)\)-approximation algorithm for submodular ones~\cite{approx_algo_for_the_maxmin_alloc_problem} have been devised in \citeyear{an_approx_algo_for_maxmin_fair_alloc_of_indiv_goods} and \citeyear{approx_algo_for_the_maxmin_alloc_problem}, respectively.
These may not be the best algorithms though, as the best known lower bound on the approximation factor is \(2\)~\cite{allocating_indiv_goods}.

In contrast, for the symmetric additive \NSW{}, an approximation hardness of \(1.069\) was shown in \citeyear{satiation_in_fisher_markets_and_approx_of_nsw}~\cite{satiation_in_fisher_markets_and_approx_of_nsw}, but only a \(1.45\)-approximation algorithm has been found in \citeyear{finding_fair_and_efficient_allocs}~\cite{finding_fair_and_efficient_allocs}.
For the symmetric submodular \NSW, an \((m - n + 1)\)-approximation algorithm has been devised in \citeyear{min_envy_and_max_avg_nsw_in_the_alloc_of_indiv_goods}~\cite{min_envy_and_max_avg_nsw_in_the_alloc_of_indiv_goods}.
However, an approximation factor dependant on the number of items is not desirable as the number of items vastly exceeds the number of agents in many applications.
Moreover, both approaches exploit the symmetry of the studied problem and fail to extend to the asymmetric case.

\Citeauthor{APNSWuSVþUM}~\cite{APNSWuSVþUM} further the research by contributing two polynomial-time algorithms for the asymmetric \NSW.
The first one, \emph{\SMatch}, computes a \(2n\)-approximative allocation under additive valuation functions.
It does so by \textsl{s}martly \textsl{match}ing agents and items which make up the parts of a bipartite graph.
The second one, \emph{\RepReMatch}, computes a \(2n (\log_2 n + 3)\)-approximative allocation under submodular valuation functions.
It does so by \textsl{rep}eatedly computing \textsl{match}ings, which then get partly annulled, so that items can be \textsl{re}matched.


\subsection{Structure of the Report}
\label{subsec:intro:structure}

We present and analyse \SMatch{} in \cref{sec:smatch} and \RepReMatch{} in \cref{sec:reprematch}.
In \cref{sec:hardness}, we give an analysis on the hardness of the submodular \NSW.
\Cref{sec:conclusion} contains the summary, an overview of newly published work since 2020, and an outlook on open questions.


\footnotetext[1]{
	The overview is given on the state of research as it was roughly at the end of the year 2019, when \citeauthor{APNSWuSVþUM} published the version of their paper~\cite{APNSWuSVþUM} on which this seminar report is based.
}