\section{Introduction}
\label{sec:intro}

\begin{itemize}
	\item
	problem introduction \& motivation

	\item
	formal problem definition

	\item
	short literature review

	\item
	results \& structure of paper
\end{itemize}

\begin{definition}
	Let \(\goods \coloneq \{1, \dots, m\}\) be a set of indivisible \emph{items} and \(\agents \coloneq \{1, \dots, n\}\) be a set of \emph{agents}.
	An \emph{allocation} is a tuple \(\alloc = ( \alloc[1], \dots, \alloc[n] ) \in \powerset[G]^n\) such that each item is element of exactly one \(\alloc[i]\), that is \(\mathop{\bigcup\hspace{-1pt}_{i \in \agents}} \alloc[i] = \goods\) and \(\alloc[i] \cap \alloc[i'] = \emptyset\) for all \(i \neq i'\).
	An item \(j \in \goods\) is \emph{assigned} to agent \(i \in \agents\) if \(j \in \alloc[i]\) holds.
\end{definition}

\(\vdots\)

\begin{definition}
	Given a set \(\goods\) of items and a set \(\agents\) of agents with \emph{valuations} \(\valuations[i] \colon \powerset[\goods] \to \real\)\todo{less strict def of valuations; restriction for our case later on} and \emph{agent weights} \(\eta_i\) for all agents \(i \in \agents\), the \emph{Nash Social Welfare problem} (\NSW) is to find an allocation maximising the weighted geometric mean of valuations, that is
	\begin{equation*}
		\argmax_{\alloc \in \allallocs{n}{\goods}} \left\{ \left( \prod_{i \in \agents} \valuations[i][\alloc[i]]^{\weight{i}} \right)^{\displaystyle 1 / \sum_{i \in \agents} \weight{i}} \right\}
	\end{equation*}
	where \(\allallocs{n}{\goods}\) is the set of all possible allocations of the items in \(\goods\) amongst \(n\) agents.
	The problem is called \emph{symmetric} if all agent weights \(\weight{i}\) are equal, and \emph{asymmetric} otherwise.
\end{definition}

\(\vdots\)

In a slight abuse of notation, we omit the brackets in a valuation function if the set of items contains only one item, that is \(\valuations[i][j] = \valuations[i][\{j\}]\).

\(\vdots\)

Garg, Kulkarni and Kulkarni consider five different types of non-negative monotonically non-decreasing valuation functions of which we are going to consider only the following two due to space constraints:
\begin{description}
	\item[Additive]
	The valuation \(\valuations[i][\genericset]\) of an agent \(i\) for a set \(\genericset \subset \goods\) of items \(j\) is the sum of individual valuations \(\valuations[i][j]\), that is \(\valuations[i][\genericset] = \sum_{j \in \genericset} \valuations[i][j]\).

	\item[Submodular]
	Let \(\valuations[i][\genericset_1 \given \genericset_2] \coloneq \valuations[i][\genericset_1 \cup \genericset_2] - \valuations[i][\genericset_2]\) denote the marginal utility of agent~\(i\) for a set \(\genericset_1 \subset \goods\) of items over the disjoint set \(\genericset_2 \subset \goods\).
	This valuation functions satisfies the submodularity constraint \(\valuations[i][j \given \genericset_1 \cup \genericset_2] \leq \valuations[i][j \given \genericset_1]\) for all agents \(i \in \agents\), items \(j \in \goods\) and sets \(\genericset_1, \genericset_2 \subset \goods\) of items.
\end{description}

\blindtext[4]