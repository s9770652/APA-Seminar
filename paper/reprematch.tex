\section{RepReMatch}
\label{sec:reprematch}

\begin{itemize}
	\item
	\SMatch{} does not work for general submodular valuations since we need to detect the set of lowest valuation.
	This is not possible independent of \(m\) [SF11].\todo{what exactly did SF11 show?}
\end{itemize}

\begin{lemma}
	\label{lem:induction}
	During phase \phaseii*{}, each agent \(i \in \agents\) gets assigned a bundle \(\alloc[\phaseii] = \{ \asgd{1}, \dots, \asgd{\alloclen[\phaseii]} \}\).
	For all rounds \(r = 2, \dots, \alloclen[\phaseii]\), the following inequality about her set \(\attopt{r}\) of optimal and attainable items holds true:
	\begin{equation*}
		\valuations[ \attopt{r} \given \asgd{1}, \dots, \asgd{r-1} ] \geq \remvalue* - \lostsetlen{2} \valuations[\asgd{1}] - \sum_{r'=2}^{r-1} \lostsetlen{r'} \cdot \valuations[ \asgd{r'} \given \asgd{1}, \dots, \asgd{r'-1} ] - \valuations[\asgd{1}, \dots, \asgd{r-1}],
	\end{equation*}
	where \(\lostsetlen{l}\) is the size of her set \(\lostset{l} \subset \attopt{l-1}\) of optimal items which were attainable in round \(l-1\) and were assigned to other agents in round \(l\), and \(\remvalue* = \valuations[\attopt{1}]\) is her valuation of attainable and optimal items during the first round.
\end{lemma}
\begin{proof}
	We prove the lemma by induction on the number \(r\) of rounds.
	In the beginning of the base case \(r=2\), agent \(i\) has already been assigned item \(\asgd{1}\) but not the items in set~\(\lostset{1}\).
	For each of the remaining optimal and attainable items~\(j\) in round \(1\), the marginal valuation \(\valuations[j \given \emptyset]\) over the empty set is at most \(\valuations[\asgd{1} \given \emptyset]\), as otherwise item~\(\asgd{1}\) would not have been assigned first.
	Furthermore, the marginal valuation \(\valuations[j \given \asgd{1}]\) over \(\{\asgd{1}\}\) is upper-bounded by \(\valuations[\asgd{1} \given \emptyset]\) due to the submodularity of valuations.
	During the round, a further \(\lostsetlen{2}\) out of these items are assigned to other agents, and item \(\asgd{2}\) is assigned to agent \(i\).
	We can bound the marginal valuation of the remaining optimal and attainable items in round \(2\) in the following way:
	\begin{caseintext}{1}{\(\asgd{1} \in \attopt{1}\)}
		It holds \(\valuations[ \attopt{2} \given \asgd{1} ] = \valuations[ \attopt{2} \cup \{\asgd{1}\} ] - \valuations[\asgd{1}] = \valuations[\attopt{1} \setminus \lostset{2}] -  \valuations[\asgd{1}]\).
	\end{caseintext}
	\begin{caseintext}{2}{\(\asgd{1} \notin \attopt{1}\)}
		Due to the monotonicity of valuations, it holds \(\valuations[ \attopt{2} \cup \{\asgd{1}\} ] \geq \valuations[ \attopt{2} ]\) and, therefore, \(\valuations[ \attopt{2} \given \asgd{1} ] \geq \valuations[ \attopt{2} ] - \valuations[\asgd{1}] = \valuations[\attopt{1} \setminus \lostset{2}] -  \valuations[\asgd{1}]\).
	\end{caseintext}
	\noindent
	In both cases, the base case is proved because
	\begin{align}
		\valuations[ \attopt{2} \given \asgd{1} ]
		&\geq \valuations[\attopt{1} \setminus \lostset{2}] -  \valuations[\asgd{1}] \\
		&\geq \valuations[ \attopt{2} ] - \valuations[\lostset{2}] - \valuations[\asgd{1}] \\
		&\geq \remvalue* - \lostsetlen{2} \valuations[\asgd{1}] - \valuations[\asgd{1}],
	\end{align}
	where the second inequality can be shown inductively with the definition of submodularity\todo{Do that or, alternatively, find a paper showing the other def.}, and the third inequality is due all \(\lostsetlen{2}\) items \(j\) in \(\lostset{2}\) being attainable but not assigned in round \(1\), implying \(\valuations[j] \leq \valuations[\asgd{1}]\).

	As induction hypothesis, we assume that the lemma is true for all rounds up to some \(r\).
	For the induction step \(r \to r+1\), we differentiate the same two cases again:
	\begin{caseintext}{1}{\(\asgd{r} \in \attopt{r}\)}
		Again we use the submodularity of valuations \todo{ditto} to obtain a lower bound on the marginal valuation of \(\attopt{r+1}\).
		\begin{align}
			\valuations[ \attopt{r+1} \given \asgd{1}, \dots, \asgd{r} ]
			&= \valuations[ \attopt{r+1} \cup \{\asgd{r}\} \given \asgd{1}, \dots, \asgd{r-1} ] - \valuations[ \asgd{r} \given \asgd{1}, \dots, \asgd{r-1} ] \\
			&= \valuations[ \attopt{r} \setminus \lostset{r+1} \given \asgd{1}, \dots, \asgd{r-1} ] - \valuations[ \asgd{r} \given \asgd{1}, \dots, \asgd{r-1} ] \\
			&\geq \begin{multlined}[t]
				\valuations[ \attopt{r} \given \asgd{1}, \dots, \asgd{r-1} ] - \valuations[ \asgd{r} \given \asgd{1}, \dots, \asgd{r-1} ] \\
				- \valuations[ \lostset{r+1} \given \asgd{1}, \dots, \asgd{r-1} ]
			\end{multlined}
		\end{align}
	\end{caseintext}
	\begin{caseintext}{2}{\(\asgd{r} \notin \attopt{r}\)}
		To obtain to the same lower bound for \(\asgd{r} \notin \attopt{r}\), we first use twice the submodularity of valuations and then the inequality
		\begin{align}
			\valuations[ \attopt{r} \given \asgd{1}, \dots, \asgd{r} ]
			&= \valuations[ \attopt{r} \cup \{ \asgd{1}, \dots, \asgd{r} \} ] - \valuations[\asgd{1}, \dots, \asgd{r}] \\
			&\geq \valuations[ \attopt{r} \cup \{ \asgd{1}, \dots, \asgd{r-1} \} ] - \valuations[\asgd{1}, \dots, \asgd{r}] \\
			&= \begin{multlined}[t]
				\paren[\big]{ \valuations[ \attopt{r} \cup \{ \asgd{1}, \dots, \asgd{r-1} \} ] - \valuations[\asgd{1}, \dots, \asgd{r-1}] } \\
				- \paren[\big]{ \valuations[\asgd{1}, \dots, \asgd{r}] - \valuations[\asgd{1}, \dots, \asgd{r-1}] }
			\end{multlined} \\
			&= \valuations[ \attopt{r} \given \asgd{1}, \dots, \asgd{r-1} ] - \valuations[ \asgd{r} \given \asgd{1}, \dots, \asgd{r-1} ]
		\end{align}
		using the monotonicity of valuations:
		\begin{align}
			\valuations[ \attopt{r+1} \given \asgd{1}, \dots, \asgd{r} ]
			&= \valuations[ \attopt{r} \setminus \lostset{r+1} \given \asgd{1}, \dots, \asgd{r} ] \\
			&\geq \valuations[ \attopt{r} \given \asgd{1}, \dots, \asgd{r} ] - \valuations[ \lostset{r+1} \given \asgd{1}, \dots, \asgd{r} ] \\
			&\geq \valuations[ \attopt{r} \given \asgd{1}, \dots, \asgd{r} ] - \valuations[ \lostset{r+1} \given \asgd{1}, \dots, \asgd{r-1} ] \\
			&\geq \begin{multlined}[t]
				\valuations[ \attopt{r} \given \asgd{1}, \dots, \asgd{r-1} ] - \valuations[ \asgd{r} \given \asgd{1}, \dots, \asgd{r-1} ] \\
				- \valuations[ \lostset{r+1} \given \asgd{1}, \dots, \asgd{r-1} ]
			\end{multlined}
		\end{align}
	\end{caseintext}
	In both cases, we can replace \(\valuations[ \attopt{r} \given \asgd{1}, \dots, \asgd{r-1} ]\) by the induction hypothesis and \(\valuations[ \lostset{r+1} \given \asgd{1}, \dots, \asgd{r-1} ]\) by \(\lostsetlen{r+1} \cdot \valuations[ \asgd{r} \given \asgd{1}, \dots, \asgd{r-1} ]\) to prove the lemma.
	For a more detailed formula simplification, we refer to Garg, Kulkarni and Kulkarni~\cite[14]{APNSWuSVþUM}.\todo[info]{\(\mathbfit{i}\): I hope the shortcut is allowed.}
\end{proof}

The lemma can be used to bound the marginal valuation of items in the round of their assignment for phase \phaseii*.
\begin{corollary}
	From \cref{lem:induction} follows
	\begin{equation*}
		\valuations[ \asgd{r} \given \asgd{1}, \dots, \asgd{r-1} ] \geq \begin{multlined}[t]
			\paren[\Big]{ \remvalue* - \lostsetlen{2} \valuations[\asgd{1}] - \sum_{r'=2}^{r-1} \lostsetlen{r'+1} \cdot \valuations[ \asgd{r'} \given \asgd{1}, \dots, \asgd{r'-1} ] \\
				- \valuations[\asgd{1}, \dots, \asgd{r-1}] } \Bigm/ \paren[\Big]{ \attoptlen{0} - \sum_{l = 1}^{r} \lostsetlen{l} }
		\end{multlined}
	\end{equation*}
	where \(\attoptlen{0} \coloneq \abs{\attopt{0}}\) denotes the number of optimal and attainable items after phase \phasei*.
\end{corollary}
\begin{proof}
	There are \(\attoptlen{0}\) optimal and attainable items in \(\attopt{0}\) at the start of phase \phaseii*.
	Of those, \(\lostsetlen{l}\) many are assigned to other agents in each round \(l \leq r\), and also some items \(\asgd{l}\) assigned to agent \(i\) may be optimal, whence an upper bound of \(\attoptlen{0} - \sum_{l=1}^{r} \lostsetlen{l}\) on the number \(\attoptlen{r}\) of items in the set \(\attopt{r}\).

	The valuations are monotone, \ie, \(\valuations[\genericset[1]] \leq \valuations[\genericset[2]]\) for all sets \(\genericset[1] \subset \genericset[2] \subset \goods\) of items.
	This implies that there must be an item \(j \in \attopt{r}\) with a marginal valuation of at least \(\valuations[ \attopt{r} \given \asgd{1}, \dots, \asgd{r-1} ] / \attoptlen{r}\).
	As item \(\asgd{r}\) was the one to be assigned, the marginal valuation of it cannot be smaller.
	Using \cref{lem:induction} for the value of \(\valuations[ \attopt{r} \given \asgd{1}, \dots, \asgd{r-1} ]\) proves the corollary.
\end{proof}

\begin{lemma}
	\begin{equation*}
		\valuations[\asgd{1}, \dots, \asgd{\alloclen[\phaseii]}] \geq \remvalue* / n
	\end{equation*}
\end{lemma}

\begin{lemma}
	about phase \phasei*
\end{lemma}

\begin{theorem}
	\label{th:reprematch}
	\RepReMatch{} has an approximation factor of \(2n (\log n + 2)\).
\end{theorem}

%\lipsum[1-29]

\begin{algorithm*}[p]
	\KwIn{%
		set \(\agents = \{ 1, \dots, n \}\) of agents with weights \(\weight \forall i \in \agents\),
		set \(\goods = \{ 1, \dots, m \}\) of indivisible items,
		submodular valuations \(\valuations \colon \powerset[\goods] \to \realposzero\) where \(\valuations[\genericset]\) is the valuation of agent \(i \in \agents\) for each set \(\genericset \subset \goods\) of items
	}
	\KwOut{%
		\(\frac{1}{2n(\log n + 2)}\)-approximation \(\alloc[\phaseiii][] = (\alloc[\phaseiii][1], \dots, \alloc[\phaseiii][n])\) of an optimal allocation
	}
	\phaseisep
	\(\alloc[\phasei] \gets \emptyset \quad\forall i \in \agents\)\;
	\(\goodsrem \gets \goods\)\;
	\For{\(t \gets 0, \dots, \ceil{\log n}-1\)}{
		\If{\(\goodsrem \neq \emptyset\)}{
			\(\weights \gets \{\,  \weight \cdot \log\paren[\big]{ \valuations[j] } \given[\bigm] i \in \agents, j \in \goods \,\}\)\;
			\(\bipartitegraph \gets (\agents, \goods, \weights)\)\;
			\(\matching \gets \maxweightmatching(\bipartitegraph)\)\;
			\(\alloc[\phasei] \gets \alloc[\phasei] \cup \{j\} \quad \forall(i, j) \in \matching\)\;
			\(\goodsrem \gets \goodsrem \setminus \{\, j \given (i, j) \in \matching \,\}\)\;
		}
	}
	\phaseiisep
	\(\alloc[\phaseii] \gets \emptyset \quad\forall i \in \agents\)   \tct*{put allocation \(\alloc[\phasei][]\) aside and start a new one}
	\While{\(\goodsrem \neq \emptyset\)}{
		\(\weights \gets \{\,  \weight \cdot \log\paren[\big]{ \valuations[ \alloc[\phaseii] \cup \{j\} ] } \given[\bigm] i \in \agents, j \in \goods \,\}\)\;
		\(\bipartitegraph \gets (\agents, \goods, \weights)\)\;
		\(\matching \gets \maxweightmatching(\bipartitegraph)\)\;
		\(\alloc[\phaseii] \gets \alloc[\phaseii] \cup \{j\} \quad \forall(i, j) \in \matching\)\;
		\(\goodsrem \gets \goodsrem \setminus \{\, j \given (i, j) \in \matching \,\}\)\;
	}
	\phaseiiisep
	\(\goodsrem \gets \mathop{\bigcup\hspace{-1pt}_{i \in \agents}} \alloc[\phasei]\)  \tct*{release items assigned in phase \phasei*}
	\(\weights \gets \{\,  \weight \cdot \log\paren[\big]{ \valuations[ \alloc[\phaseii] \cup \{j\} ] } \given[\bigm] i \in \agents, j \in \goods \,\}\)\;
	\(\bipartitegraph \gets (\agents, \goods, \weights)\)\;
	\(\matching \gets \maxweightmatching(\bipartitegraph)\)\;
	\(\alloc[\phaseiii] \gets \alloc[\phaseii] \cup \{j\} \quad\forall(i, j) \in \matching\)\;
	\(\goodsrem \gets \goodsrem \setminus \{\, j \given (i, j) \in \matching \,\}\)\;
	\(\alloc[\phaseiii] \gets \arballoc( \agents, \goodsrem, \alloc[\phaseiii], (\valuations)_{i \in \agents} )\)
	\todo{decide on \(\alloc[][]\) or \(\alloc[\phaseiii][]\); discrepancies in orig paper in def of \(\alloc[\phaseiii][]\)!}\;
	\Return{\(\alloc[\phaseiii][]\)}
	\caption{%
		\RepReMatch{} for the Asymmetric Submodular \NSW{} problem
	}
	\label{alg:reprematch}
\end{algorithm*}