\section{RepReMatch}
\label{sec:reprematch}

\subsection{Presentation of the Algorithm}
\label{subsec:reprematch:presentation}

The algorithm \SMatch{} estimates the worst-case valuation of future items by determining the set of highest-value items and then valuing the remaining items.
Unfortunately, this approach does not work for general \emph{submodular} valuation functions because taking a set of highest valuation away does not necessarily leave a set of lowest valuation.
In fact, it can be shown \cite{submodular_low_value} that determining the set of lowest valuation is approximable only within a factor of \(\smash[t]{\bigomega*{ \sqrt{m / \ln m} }}\).

For this reason, the algorithm \RepReMatch{} (\cref{alg:reprematch}) relies on an approach with three phases, thereby achieving an approximation factor of \(2n(\log_2 n + 3)\) (\cref{th:reprematch}).
In phase~\phasei, a sufficiently big set of high-value items is determined through repeated matchings.
This phase serves merley to determine this set, so items are assigned temporarily only.
The edge weights reflect this by taking the valuations of just single items into account.
\vfil
\begin{algorithm*}[hb]
	\KwIn{%
		a set \(\goods\) of \(m\) items,
		a set \(\agents\) of \(n\) agents,
		submodular valuation functions \(\valuations \colon \powerset{\goods} \to \realposzero\) and weights \(\weight \in \realpos\) for all agents \(i \in \agents\)
	}
	\KwOut{%
		a \(2n(\log_2 n + 3)\)-approximation \(\alloc[\phaseiii][] = (\alloc[\phaseiii])_{i \in \agents}\) of an optimal allocation
	}
	\phaseisep
	\(\alloc[\phasei] \gets \emptyset \quad\forall i \in \agents\)  \tct*{initialise temporary allocation}
	\(\goodsrem \gets \goods\)  \tct*{set of unassigned items}
	\For{\(t \gets 1, \dots, \ceil{\log_2 n}+1\)}{
		\If{\(\goodsrem \neq \emptyset\)}{
			\(\weights \gets \braces[\big]{\, \paren[\big]{ i, j, \weight \log{ \valuations[ \hairspace j ] } } \given[\bigm] i \in \agents, j \in \goodsrem \,}\)   \tct*{weighted edges using val. of single item}
			\(\bipartitegraph \gets (\agents, \goodsrem, \weights)\)  \tct*{bipartite graph}
			\(\matching \gets \maxweightmatching(\bipartitegraph)\)\;
			\(\alloc[\phasei] \gets \alloc[\phasei] \cup \{ \hairspace j \} \quad \forall(i, j) \in \matching\)  \tct*{assign (temporarily) according to matching}
			\(\goodsrem \gets \goodsrem \setminus \{\, \hairspace j \given (i, j) \in \matching \,\}\)  \tct*{remove assigned items}
		}
	}
	\phaseiisep
	\(\alloc[\phaseii] \gets \emptyset \quad\forall i \in \agents\)   \tct*{put allocation \(\alloc[\phasei][]\) away \& initialise new, definite one}
	\While{\(\goodsrem \neq \emptyset\)}{
		\(\weights \gets \braces[\big]{\, \paren[\big]{ i, j, \weight \log{ \valuations[ \alloc[\phaseii] \cup \{ \hairspace j \} ] } } \given[\bigm] i \in \agents, j \in \goodsrem \,}\)   \tct*{valuation of item \& current bundle}
		\(\bipartitegraph \gets (\agents, \goodsrem, \weights)\)\;
		\(\matching \gets \maxweightmatching(\bipartitegraph)\)\;
		\(\alloc[\phaseii] \gets \alloc[\phaseii] \cup \{ \hairspace j \} \quad \forall(i, j) \in \matching\)  \tct*{assign (definitely) according to matching}
		\(\goodsrem \gets \goodsrem \setminus \{\, \hairspace j \given (i, j) \in \matching \,\}\)\;
	}
	\phaseiiisep
	\(\goodsrem \gets \bigcup_{i \in \agents} \alloc[\phasei]\)  \tct*{release items assigned in phase~\phasei} \label{ln:goodsrem}
	\(\weights \gets \braces[\big]{\, \paren[\big]{ i, j, \weight \log{ \valuations[ \alloc[\phaseii] \cup \{ \hairspace j \} ] } } \given[\bigm] i \in \agents, j \in \goodsrem \,}\)   \tct*{valuation of item \& current bundle}
	\(\bipartitegraph \gets (\agents, \goodsrem, \weights)\)\;
	\(\matching \gets \maxweightmatching(\bipartitegraph)\)\;
	\(\alloc[\phaseiii] \gets \alloc[\phaseii] \cup \{ \hairspace j \} \quad\forall(i, j) \in \matching\)  \tct*{initialise final allocation \& assign according to matching}
	\(\goodsrem \gets \goodsrem \setminus \{\, \hairspace j \given (i, j) \in \matching \,\}\)\;
	\(\alloc[\phaseiii][] \gets \arballoc( \agents, \goodsrem, \alloc[\phaseiii][], (\valuations)_{i \in \agents} )\)  \tct*{assign remainder of items arbitrarily}
	\Return{\(\alloc[\phaseiii][]\)}
	\caption{%
		\RepReMatch{} for the asymmetric submodular \NSW
	}
	\label{alg:reprematch}
\end{algorithm*}

In phase~\phaseii, the remaining items are assigned definitely through repeated matchings.
Consequently, each edge weight is updated in each round to reflect the valuation of both the respective item and the items assigned so far.

In phase~\phaseiii, the high-value items assigned in phase \phasei{} are released.
One maximum weight matching is calculated, and the matched items are assigned accordingly.
Similarly to the previous phase, each edge weight uses the valuation of both the respective item and the items assigned so far, \ie, the bundle from phase \phaseii.
The remainder of released items is assigned arbitrarily.

\subsection{Analysis of the Algorithm}
\label{subsec:reprematch:analysis}

We use the term \emph{round} to refer to the iterations of the loops in the phases~\phasei{} and~\phaseii.
For ease of notation, we refer to the moment right before the first iteration in phase~\phaseii{} as round \(0\).

We start by analysing phase~\phaseii{} as it is the first phase with definitive assignments.
To this end, we introduce two types of item sets.
\begin{definition}
	Let \(\alloc*\) be an optimal bundle of an agent \(i \in \agents\).
	For any round \(r \ge 1\) in phase~\phaseii, the set \(\lostset{r} \subset \alloc*\) of \emph{lost items} is the set of all items \(j \in \alloc*\) assigned to other agents \(i' \neq i\) in that round.
\end{definition}
\begin{definition}
	Let \(\alloc*\) be an optimal bundle of an agent \(i \in \agents\) and \(\alloc[\phaseii] = \{ \asgd{1}, \dots, \asgd{\alloclen[\phaseii]} \}\) be her bundle at the end of phase \phaseii.
	The set of \emph{optimal and attainable items} is defined as \(\attopt{0} \coloneq \alloc* \setminus \bigcup_{i' \in \agents} \alloc[\phasei][i']\) for round~\(0\), as \(\attopt{1} \coloneq \attopt{0} \setminus \lostset{1}\) for round \(1\) and as \(\attopt{r} \coloneq \attopt{r-1} \setminus ( \lostset{r} \cup \{\asgd{r-1}\} )\) for round \(r = 2, \dots, \alloclen[\phaseii]\) of phase \phaseii.
\end{definition}
\noindent
We denote their sizes by \(\lostsetlen{r}\ \coloneq \abs{\lostset{r}}\) and \(\attoptlen{r} \coloneq \abs{\attopt{r}}\), respectively.
Firstly, we give a lower bound on the valuations of optimal and attainable items.
\begin{lemma}
	\label{lem:induction}
	For each agent \(i \in \agents\) and her bundle \(\alloc[\phaseii] = \{ \asgd{1}, \dots, \asgd{\alloclen[\phaseii]} \}\) at the end of phase \phaseii, it holds in all rounds \(r = 2, \dots, \alloclen[\phaseii]\) that
	\vspace{-2ex}
	\begin{equation*}
		\valuations[ \attopt{r} \given \asgd{1}, \dots, \asgd{r-1} ] \ge \valuations[\attopt{1}] - \smashoperator{\sum_{r'=1}^{r-1}} \lostsetlen{r'+1} \cdot \valuations[ \asgd{r'} \given \asgd{1}, \dots, \asgd{r'-1} ] - \valuations[\asgd{1}, \dots, \asgd{r-1}] \mperiod \vspace{-0.5ex}
	\end{equation*}
\end{lemma}
\begin{proof}
	The original proof~\cite[13\psq]{APNSWuSVþUM} consists of a lengthy induction rich in formulae and case differentiations.
	We opt to give a different but more intuitive approach.

	Writing the left-hand side of the lemma out in full gives
	\begin{equation}
		\valuations[ \attopt{r} \given \asgd{1}, \dots, \asgd{r-1} ]
		= \valuations[ \attopt{r} \cup \braces{ \asgd{1}, \dots, \asgd{r-1} } ] - \valuations[ \asgd{1}, \dots, \asgd{r-1} ] \mperiod[,]
		\label{eq:attopt_marginal}
	\end{equation}
	which explains the last subtrahend on the right-hand side.

	Next, we show a lower bound on \(\valuations[\attopt{r} \cup \braces{ \asgd{1}, \dots, \asgd{r-1} }]\).
	The items which are optimal and attainable in round \(r\) were so in round~\(r-1\), too.
	Additionally, the optimal but lost items of round~\(r\) were attainable in round \(r-1\) as well.
	The item \(\asgd{r-1}\) assigned to agent \(i\) in round \(r-1\) may also be optimal.
	Therefore, it holds \(\paren{ \attopt{r} \cup \braces{\asgd{r-1}} } \supset \paren{ \attopt{r-1} \setminus \lostset{r} }\) and
	\begin{align}
		\valuations[\attopt{r} \cup \braces{ \asgd{1}, \dots, \asgd{r-1} }]
		&\ge \valuations[\attopt{r-1} \setminus \lostset{r} \cup \braces{ \asgd{1}, \dots, \asgd{r-2} }] \\
		&\ge \valuations[\attopt{r-1} \cup \braces{ \asgd{1}, \dots, \asgd{r-2} }] - \valuations[ \lostset{r} \given \asgd{1}, \dots, \asgd{r-2} ] \mperiod
		\label{eq:attopt_union}
	\end{align}
	The second inequality stems from the property of diminishing returns.
	If we now recursively apply \cref{eq:attopt_union}, we eventually arrive at
	\vspace{-0.5ex}
	\begin{equation}
		\valuations[\attopt{r} \cup \braces{ \asgd{1}, \dots, \asgd{r-1} }]
		\ge \valuations[\attopt{1}] - \smashoperator{\sum_{r'=1}^{r-1}} \valuations[ \lostset{r'+1} \given \asgd{1}, \dots, \asgd{r'-1} ] \mperiod
		\label{eq:attop_lost}
	\end{equation}

	In the last step, we give an upper bound on \(\valuations[ \lostset{r'+1} \given \asgd{1}, \dots, \asgd{r'-1} ]\).
	The valuation \(\valuations[\hairspace j \given \asgd{1}, \dots, \asgd{r'-1}]\) of any item \(j \in \lostset{r'+1}\) in round \(r'\) could not have been higher than \(\valuations[\asgd{r'} \given \asgd{1}, \dots, \asgd{r'-1}]\) as otherwise item \(\asgd{r'}\) would not have been assigned before item \(j\).
	From this follows
	\begin{equation}
		\valuations[ \lostset{r'+1} \given \asgd{1}, \dots, \asgd{r'-1} ] \le \lostsetlen{r'+1} \cdot \valuations[\asgd{r'} \given \asgd{1}, \dots, \asgd{r'-1}] \mperiod
		\label{eq:lost}
	\end{equation}
	Combining \cref{eq:attopt_marginal,eq:attop_lost,eq:lost} proves the lemma.
\end{proof}

The lemma can be used to find a lower bound on the marginal valuation of the items actually assigned in each round \(r\).
\begin{corollary}
	\label{cor:lower_bound_single_item}
	From \cref{lem:induction} follows
	\begin{equation*}
		\valuations[ \asgd{r} \given \asgd{1}, \dots, \asgd{r-1} ]
		\ge \paren[\Big]{ \valuations[\attopt{1}] - \smashoperator{\sum_{r'=1}^{r-1}} \lostsetlen{r'+1} \cdot \valuations[ \asgd{r'} \given \asgd{1}, \dots, \asgd{r'-1} ] - \valuations[\asgd{1}, \dots, \asgd{r-1}] } \Big/ \attoptlen{r} \mperiod
	\end{equation*}
\end{corollary}
\begin{proof}
	Valuation functions are monotonic, so \(\valuations[\genericset[1]] \le \valuations[\genericset[2]]\) holds for all sets \(\genericset[1] \subset \genericset[2] \subset \goods\).
	Induction shows that there must be an item \(j \in \genericset[2]\) with valuation \(\valuations[\hairspace j] \ge \valuations[\genericset[2]] / \abs{\genericset[2]}\), otherwise it would hold \(\valuations[\emptyset] > 0\).
	Applied to \cref{lem:induction}, this means that there must be an item \(j \in \attopt{r}\) with a marginal valuation of at least \(\valuations[ \attopt{r} \given \asgd{1}, \dots, \asgd{r-1} ] / \attoptlen{r}\)\mperiod{}
	As item \(\asgd{r}\) was the one to be assigned in round \(r\), its marginal valuation cannot be smaller.
\end{proof}

This, finally, enables us to give a lower bound on the valuation of the bundles \(\alloc[\phaseii]\).
\begin{lemma}
	\label{lem:lower_bound_all_items}
	For each agent \(i \in \agents\) and her bundle \(\alloc[\phaseii] = \{\asgd{1}, \dots, \asgd{\alloclen[\phaseii]}\}\), it holds
	\begin{equation*}
		\valuations[\asgd{1}, \dots, \asgd{\alloclen[\phaseii]}] \ge \valuations[\attopt{1}] / n \mperiod
	\end{equation*}
\end{lemma}
\begin{proof}
	In each round \(r = 1, \dots, \alloclen[\phaseii]\), there are \(\lostsetlen{r}\) optimal and attainable items of agent \(i\) gettin assigned to other agents.
	As \(n\) is the number of agents, \(n-1\) is an upper bound on \(\lostsetlen{r}\).
	Furthermore, after \(\alloclen[\phaseii]\) rounds, the number of unassigned items is at most \(n-1\) elsewise agent~\(i\) would have been assigned yet another item in a round \(\alloclen[\phaseii] + 1\).
	This implies \(\attoptlen{\alloclen[\phaseii]} \le n\), as those \(n-1\) items could be optimal and item \(\asgd{\alloclen[\phaseii]}\) may be, too.
	Together with \cref{cor:lower_bound_single_item}, the lemma is proven:
	\begin{align}
		\valuations[\asgd{1}, \dots, \asgd{\alloclen[\phaseii]}]
		&= \valuations[\asgd{1}, \dots, \asgd{\alloclen[\phaseii] - 1}] + \valuations[ \asgd{\alloclen[\phaseii]} \given \asgd{1}, \dots, \asgd{\alloclen[\phaseii] - 1} ] \\
		&\ge \begin{multlined}[t]
			\valuations[\asgd{1}, \dots, \asgd{\alloclen[\phaseii] - 1}] + \paren[\Big]{ \valuations[\attopt{1}] - \smashoperator{\sum_{r'=1}^{\alloclen[\phaseii]-1 \mathstrut}} \lostsetlen{r'+1} \cdot \valuations[ \asgd{r'} \given \asgd{1}, \dots, \asgd{r'-1} ] \\
				- \valuations[\asgd{1}, \dots, \asgd{\alloclen[\phaseii]-1}] } \Big/ \attoptlen{\alloclen[\phaseii]}  % \Big/ has better spacing than \Bigm/.
		\end{multlined} \\
		&\ge \begin{multlined}[t]
			\valuations[\asgd{1}, \dots, \asgd{\alloclen[\phaseii] - 1}] + \paren[\Big]{ \valuations[\attopt{1}] - \smashoperator{\sum_{r'=1}^{\alloclen[\phaseii]-1 \mathstrut}} (n-1) \valuations[ \asgd{r'} \given \asgd{1}, \dots, \asgd{r'-1} ] \\
				- \valuations[\asgd{1}, \dots, \asgd{\alloclen[\phaseii]-1}] } \Big/ n
		\end{multlined} \\
		&\ge \valuations[\asgd{1}, \dots, \asgd{\alloclen[\phaseii] - 1}] + \paren[\big]{ \valuations[\attopt{1}] - (n-1) \valuations[\asgd{1}, \dots, \asgd{\alloclen[\phaseii]-1}] - \valuations[\asgd{1}, \dots, \asgd{\alloclen[\phaseii]-1}] } \big/ n \\
		&= \valuations[\attopt{1}] / n
	\end{align}
\end{proof}

After having obtained a lower bound on the valuation of items assigned in phase~\phaseii, we need a lower bound for phase~\phaseiii{} as well.
Therefor we introduce a third type of item set.
\begin{definition}
	Let \(\alloc* = \{ \asgd*{1}, \dots, \asgd*{\alloclen*}\}\) be an optimal bundle of an agent \(i \in \agents\).
	The set of \emph{outstanding items} is defined as \(\overlygoodset \coloneq \{\, \hairspace j \in \goods \given \valuations[\hairspace j] \ge \valuations[\asgd*{1}] \,\}\).
\end{definition}

\begin{lemma}
	\label{lem:overly_good_matching}
	In phase~\phaseiii, there exists a matching such that each agent \(i \in \agents\) is matched to one of her outstanding items in the set \(\bigcup_{i' \in \agents} \alloc[\phasei][i']\) of released items.
\end{lemma}
\begin{proof}
	If all items were matched in phase~\phasei, \ie, \(\bigcup_{i' \in \agents} \alloc[\phasei][i'] = \goods\), then all optimal items are released in phase~\phaseiii{} and each agent can be matched to one;
	the lemma is proven immediately.
	If not, imagine for some~\(t\) that only the items assigned in the first~\(t\) rounds of phase~\phasei{} were released.
%	Denote the set of released items by \(\goodsreleased{t}\).
	Now choose some matching \(\matching[t]\) with the following properties:
	\begin{enumerate}
		\item
		\label[property]{enum:matching:agent}
		If all outstanding items of an agent \(i\) were amongst the released items, she gets matched with an outstanding item \(j \in \overlygoodset\).

		\item
		\label[property]{enum:matching:max}
		The number of agents matched with one of their outstanding items is maximal amongst all matchings fulfilling \cref{enum:matching:agent}.
	\end{enumerate}
	\Cref{enum:matching:agent} is always satisfiable as the union of \(k\) many sets \(\overlygoodset\) contains \(k\) different outstanding items~\(\asgd*{1}\), which can be matched with agents \(i\).
	\Cref{enum:matching:max} leads to all agents being matched with an outstanding item when \(t = \ceil{\log_2 n}+1\), \ie{} the number of rounds in phase~\phasei, whence the lemma follows.
	To prove this, we denote by \(\unluckyagents{t}\) the set of agents who are \emph{not} matched with one of their outstanding items, and show by induction on \(t\) that it holds \(\abs{\unluckyagents{t}} \le n/2^t\).

	In the base case \(t=1\), none of the items are assigned initially.
	Denote by \(\unluckyagentsalgo{1}\) the number of agents who were not assigned an outstanding item in the first round of phase~\phasei.
	If \(\unluckyagentsalgo{1} \le n/2\), then a matching \(\matching[1]\) as described above obviously exists and the base case is immediately proven.
	Otherwise, all items from at least \(\unluckyagentsalgo{1}\) many sets \(\overlygoodset\) got assigned to someone.
	Again:
	Each set \(\overlygoodset\) contains the outstanding item \(\asgd*{1}\), so the union of these sets contains at least~\(\unluckyagentsalgo{1}\) items which can be matched with at least \(\unluckyagentsalgo{1}\) agents upon release.
	This, then, leaves at most~\(n-\alpha < n/2\) agents not matched with an outstanding item.

	For the induction hypothesis, we assume that the statement holds true for all rounds up to some \(t\).
	In the induction step \(t \to t+1\), by \cref{enum:matching:agent}, there is at least one unassigned, outstanding item in each set \(\overlygoodset[i']\) of all agents \(i' \in \unluckyagents{t}\) at the start of round \(t+1\).
	Analogously to the base case, for at least half of those agents \(i'\), these unassigned items will be assigned to either them or someone else, and it can be argued accordingly.
	By the induction hypothesis, it holds \(\abs{\unluckyagents{t+1}} \le \abs{\unluckyagents{t}}/2 \le (n/2^t)/2 = n/2^{t+1}\).
\end{proof}

This allows us to calculate an approximation factor for \RepReMatch{} by comparing its output with an optimal allocation \(\alloc*[][]\).
\begin{theorem}
	\label{th:reprematch}
	\RepReMatch{} has an approximation factor of \(2n (\log_2 n + 3)\).
\end{theorem}
\begin{proof}
	By \cref{lem:overly_good_matching}, we can assign each agent \(i\) an outstanding item \(\overlygooditem \in \overlygoodset\) in the beginning of phase~\phaseiii.
	\RepReMatch{} maximises the logarithmic Nash social welfare, so
	\begin{equation}
		\label{eq:reprematch_approx_factor_lower_bound}
		\log \NSW(\alloc[\phaseiii][])
		\ge \frac{1}{\sum_{i \in \agents} \weight} \cdot \sum_{i \in \agents} \weight \log \valuations[\hairspace \overlygooditem, \asgd{1}, \dots, \asgd{\alloclen[\phaseii]}]
	\end{equation}
	is a lower bound on the logarithmic Nash social welfare after the first matching in phase~\phaseiii, with \(\alloc[\phaseii] = \{ \asgd{1}, \dots, \allowbreak \asgd{\alloclen[\phaseii]} \}\) being the bundle of agent \(i\) at the end of phase~\phaseii.

	\Cref{lem:lower_bound_all_items} delivers a straightforward lower bound on the valuations of bundles, namely
	\begin{equation}
		\label{eq:assigned_items_lower_bound}
		\valuations[ \hairspace \overlygooditem, \asgd{1}, \dots, \asgd{\alloclen[\phaseii]} ]
		\ge \valuations[ \asgd{1}, \dots, \asgd{\alloclen[\phaseii]} ]
		\ge \frac{\valuations[\attopt{1}]}{n}
		= \frac{\valuations[ \attopt{0} \setminus \lostset{1} ]}{n} \mperiod
	\end{equation}
	Next, consider the complement of \(\attopt{0} \setminus \lostset{1}\), \ie{} \((\alloc* \setminus \attopt{0}) \cup \lostset{1}\).
	Phase~\phasei{} runs for at most \(\ceil{\log_2 n}+1\) rounds, and at most~\(n\) items are assigned in each iteration.
	Therefore, at most \(n (\log_2 n + 2)\) optimal items are assigned in that phase, \ie, \(\abs{\bigcup_{i' \in \agents} \alloc[\phasei][i']} = \abs{ \alloc* \setminus \attopt{0} } \le n (\log_2 n + 2)\).
	As in \cref{lem:lower_bound_all_items}, it also holds \(n \ge \lostsetlen{1} = \abs{\lostset{1}}\).
	Since \({ (\alloc* \setminus \attopt{0}) \cup \lostset{1} } \subset \alloc*\) and \(\valuations[\asgd*{1}] \ge \valuations[\asgd*{r}]\) for all \(\asgd*{r} \in \alloc*\), it holds
	\begin{equation}
		\label{eq:overly_good_item_lower_bound}
		\valuations[ \hairspace \overlygooditem, \asgd{1}, \dots, \asgd{\alloclen[\phaseii]} ]
		\ge \valuations[\hairspace \overlygooditem]
		\ge \valuations[\asgd*{1}]
		\ge \frac{\valuations[ (\alloc* \setminus \attopt{0}) \cup \lostset{1} ][\big]}{n(\log_2 n + 3)} \mperiod
	\end{equation}
	\Cref{eq:overly_good_item_lower_bound,eq:assigned_items_lower_bound}, combined with the submodularity of the valuation functions, give the concise lower bound
	\vspace{-1.5ex}
	\begin{align}
		\valuations[ \hairspace \overlygooditem, \asgd{1}, \dots, \asgd{\alloclen[\phaseii]} ]
		&\ge \frac{1}{2} \paren*{ \frac{\valuations[ (\alloc* \setminus \attopt{0}) \cup \lostset{1} ][\big]}{n(\log_2 n + 3)} + \frac{\valuations[ \attopt{0} \setminus \lostset{1} ][\big]}{n} } \\
		&\ge \frac{1}{2} \cdot \frac{\valuations[ (\alloc* \setminus \attopt{0}) \cup \lostset{1} ][\big] + \valuations[ \attopt{0} \setminus \lostset{1} ][\big]}{n(\log_2 n +3)} \\
		&\ge \frac{1}{2} \cdot \frac{ \valuations[ \paren{(\alloc* \setminus \attopt{0}) \cup \lostset{1}} \cup \paren{\attopt{0} \setminus \lostset{1}} ][\big] }{n(\log_2 n + 3)} \\
		&= \frac{ \valuations[\alloc*] }{2n(\log_2 n + 3)} \mperiod
	\end{align}
	We can insert this lower bound into \cref{eq:reprematch_approx_factor_lower_bound} and prove the theorem thereby:
	\begin{equation}
		\log \NSW(\alloc[\phaseiii][])
		\ge \frac{1}{\sum_{i \in \agents} \weight} \cdot \sum_{i \in \agents} \weight \log \paren[\bigg]{ \frac{ \valuations[\alloc*] }{2n(\log_2 n + 3)} }
		= \log \paren[\bigg]{ \frac{\NSW(\alloc*[][])}{2n(\log_2 n + 3)} }
	\end{equation}
\end{proof}