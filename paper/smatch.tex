\section{\texorpdfstring{\SMatch}{SMatch}}
\label{sec:smatch}

In the case of an equal number of agents and items, \ie, \(n = m\), the \emph{additive} \NSW{} can be solved exactly by finding a maximum matching on a bipartite graph with the sets of agents and of items as its parts;
as weight of the edge between agent \(i\) and item \(j\), use \(\weight \log \valuations[j]\), that is the weighted valuation of item \(j\) by agent \(i\) in the \emph{logarithmic} Nash social welfare.
Should there be more items than agents, then it would be obvious at first to just repeatedly find a maximum matching and assign the items accordingly until all items are assigned.
The flaw of this idea is that such a greedy algorithm only considers the valuations of items in the current matching and perhaps the valuations of items already assigned.
As the example in \cref{fig:allocation} demonstrates, this leads to an algorithm with an approximation factor dependent on the number \(m\) of items.\todo{What is the factor?}
The geometric mean of the \NSW{} favours allocations with similarly valued bundles, wherefore it may be beneficial to give items to agents who cannot expect many more valuable items in the future instead of to agents who value the item a bit more but do so for other items as well.

\begin{figure*}
	\tikzset{
		lbl/.style = {font=\small},
		agent/.style = {draw, circle, font=\small},
		item/.style = {draw, rectangle, font=\small},
		dots/.style = {font=\small},
		weight/.style = {font=\footnotesize},
		edgeopt/.style = {},
		edgealg/.style = {},
		edgegry/.style = {},
		node distance=7mm and 20mm,
	}
	\def\allocationexample{
		\begin{tikzpicture}
			% items
			\node[item, label={[lbl]above:\(\goods\)}] (i1)                       {1};
			\node[item]                                (i2)  [below=of i1]        {2};
			\node[item]                                (i3)  [below=of i2]        {3};
			\node[item]                                (im)  [below=of i3]        {\(m\)};
			\node[item]                                (im1) [below=of im]        {\(m+1\)};
			\node[dots]                                (id)  at ($(i3)!0.5!(im)$) {\(\vdots\)};
			% agents
			\node[agent, label={[lbl]above:\(\agents\)}] (a1) [left=of i1]  {1};
			\node[agent]                                 (a2) [left=of im1] {2};
			% valuations
			\draw[edgealg] (a1) to node[weight, pos=.43, below]      {\(m + \epsilon\)} (i1);
			\draw          (a1) to node[weight, pos=.42, below]      {\(1\)} (i2);
			\draw          (a1) to node[weight, pos=.37, below]      {\(1\)} (i3);
			\draw          (a1) to node[weight, pos=.30, below]      {\(1\)} (im);
			\draw[edgeopt] (a1) to node[weight, pos=.26, below left] {\(1\)} (im1);

			\draw[edgeopt] (a2) to node[weight, pos=.26, above left] {\(m\)} (i1);
			\draw[edgegry] (a2) to node[weight, pos=.30, above]      {\(0\)} (i2);
			\draw[edgegry] (a2) to node[weight, pos=.37, above]      {\(0\)} (i3);
			\draw[edgegry] (a2) to node[weight, pos=.42, above]      {\(0\)} (im);
			\draw[edgealg] (a2) to node[weight, pos=.48, above]      {\(1\)} (im1);
		\end{tikzpicture}
	}
	\centering
	\begin{subfigure}{0.325\textwidth}
		\allocationexample
		\caption{%
			problem instance
		}
	\end{subfigure}
	\hfill
	\begin{subfigure}{0.325\textwidth}
		\tikzset{
			edgealg/.style = {white},
			edgegry/.style = {white},
		}
		\allocationexample
		\caption{%
			\(\OPT = \sqrt{m \cdot m}\)
		}
	\end{subfigure}
	\hfill
	\begin{subfigure}{0.325\textwidth}
		\tikzset{
			edgeopt/.style = {white},
			edgegry/.style = {white},
		}
		\allocationexample
		\caption{%
			\(\ALG \le \sqrt{(2m + \epsilon - 1) \cdot 1}\)
		}
	\end{subfigure}
	\caption{%
		An example showing that simple repeated matching without consideration of the future leads to an approximation factor dependent on the number of items.
		Agent \(1\) values item \(1\) at \(m + \epsilon\) and all other items at \(1\).
		Agent \(2\) values item \(1\) at \(m\), item \(m+1\) at~\(1\) and all other items at \(0\).
		In an optimal allocation, item \(1\) would be assigned to agent \(2\) and all other items to agent \(1\), resulting in a NSW of \(\sqrt{m \cdot m} = m\).
		A repeated maximum matching algorithm would greedily assign item \(1\) to agent \(1\) and item \(m+1\) to agent \(2\) in the first round.
		Even if all remaining items were going to be assigned to agent \(1\), the NSW will never surpass \(\sqrt{(2m + \epsilon - 1) \cdot 1} < \sqrt{2m}\).
		The approximation factor \(\alpha \approx \sqrt{m/2}\) therefore depends on the number of items.
		\todo{fix pos of subcaptions \& of agent 2}
	}
	\label{fig:allocation}
\end{figure*}

The algorithm \SMatch{}, described in \cref{alg:smatch}, eliminates the flaw by first gaining foresight of the valuations of items assigned after the first matching, achieving an approximation factor of \(2 n\) (cf.\@ \cref{th:smatch} later on).
For a fixed agent \(i\), order the items in descending order of valuations and denote the \(j\)-th most liked item by~\(\goodsordered{j}\).
To obtain a well-defined order, items of equal rank are further ordered numerically.
\SMatch{} does in fact repeatedly match items.
During the first matching, however, the edge weights are defined as \(\weight \log\paren[\big]{ \valuations[j] + \remvalue/n } \) for an edge between agent \(i\) and item \(j\).
The addend \(\remvalue\) serves as estimation of the valuation of items assigned after the first matching (cf. \cref{lem:lower_bound_later_items}) and is defined as
\begin{equation}
	\label{eq:def_remvalue}
	\remvalue
	\coloneq \smashoperator{\min_{\substack{\genericset \subset \goods \\ \abs{\genericset} \le 2n}}} \{ \valuations[\goods \setminus \genericset] \}
	= \valuations[ \goodsordered{2n+1}, \dots, \goodsordered{m} ].
\end{equation}
The set \(\genericset\) has less than \(2n\) elements only if there are less than \(2n\) items in total.
From the second matching onwards, the edge weights are defined as \(\weight \log\paren[\big]{ \valuations[j] + \valuations[\alloc] }\), where~\(\alloc\) is the continuously updated bundle of agent \(i\).
The addend \(\valuations[\alloc]\) could lead to better allocations in applications, but does not improve the approximation factor asymptotically.

\begin{algorithm*}[t]
	\KwIn{%
		set \(\goods\) of \(m\) items,
		set \(\agents\) of \(n\) agents, additive valuation functions \(\valuations \colon \powerset[\goods] \to \realposzero\) and weights \(\weight\) for all agents \(i \in \agents\)
	}
	\KwOut{%
		\(\frac{1}{2n}\)-approximation \(\alloc[][] = (\alloc)_{i \in \agents}\) of an optimal allocation
	}

	\(\alloc \gets \emptyset \quad\forall i \in \agents\)\;
	\(\remvalue \gets \valuations[ \goodsordered{2n+1}, \dots, \goodsordered{m} ] \quad\forall i \in \agents\)  \tct*{estimation of future valuations}
	\(\weights \gets \{\,  \weight \cdot \log\paren[\big]{ \valuations[j] + \remvalue/n } \given[\bigm] i \in \agents, j \in \goods \,\}\)  \tct*{edge weights}
	\(\bipartitegraph \gets (\agents, \goods, \weights)\)  \tct*{bipartite graph}
	\(\matching \gets \maxweightmatching(\bipartitegraph)\)\;
	\(\alloc \gets \{\, j \given (i, j) \in \matching \,\} \quad\forall i \in \agents\)  \tct*{assign according to matching}
	\(\goodsrem \gets \goods \setminus \{\, j \given (i, j) \in \matching \,\}\)  \tct*{remove assigned items}
	\While{\(\goodsrem \neq \emptyset\)}{
		\(\weights \gets \{\,  \weight \cdot \log\paren[\big]{ \valuations[j] + \valuations[\alloc] } \given[\bigm] i \in \agents, j \in \goodsrem \,\}\)\;
		\(\bipartitegraph \gets (\agents, \goodsrem, \weights)\)\;
		\(\matching \gets \maxweightmatching(\bipartitegraph)\)\;
		\(\alloc \gets \alloc \cup \{\, j \given (i, j) \in \matching \,\} \quad\forall i \in \agents\)\;
		\(\goodsrem \gets \goodsrem \setminus \{\, j \given (i, j) \in \matching \,\}\)\;
	}
	\Return{\(\alloc[][]\)}
	\caption{%
		\SMatch{} for the Asymmetric Additive \NSW{} problem
	}
	\label{alg:smatch}
\end{algorithm*}

To calculate the approximation factor of \SMatch, we first need to establish a lower bound on the valuation of single items.
For convenience, we order the items in the final bundle \(\alloc = \{\asgd{1}, \dots, \asgd{\alloclen}\}\) of agent \(i\) by the order in which they were assigned, so that item \(\asgd{t}\) is assigned according to the \(t\)-th matching.
Note that it holds \(\valuations[\asgd{t}] \ge \valuations[\asgd{t'}]\) for all \(t' \ge t\).
\begin{lemma}
	\label{lem:lower_bound_single_item}
	For each agent \(i \in \agents\), her final bundle \(\alloc = \{\asgd{1}, \dots, \asgd{\alloclen}\}\), and her \(tn\)-th most highly valued item \(\goodsordered{tn}\), it holds \(\valuations[\asgd{t}] \ge \valuations[\goodsordered{tn}]\) for all \(t = 1, \dots, \alloclen\).
\end{lemma}
\begin{proof}
	Before the \(t\)-th matching, no more than \((t-1) n\) items out of the \(tn\) most highly valued items \(\goodsordered{1}, \dots, \goodsordered{tn}\) have been assigned in previous matchings since at most~\(n\) many out of those items are assigned each time.
	Because of the \(t\)-th matching, at most \(n-1\) more could be assigned to all other agents \(i' \neq i\), leaving at least one item of \(\goodsordered{1}, \dots, \goodsordered{tn}\) unassigned.
	Since \(\valuations[\goodsordered{k}] \ge \valuations[\goodsordered{tn}]\) for all \(k \le tn\) by definition of \(\goodsordered{\cdot}\), the lemma follows.
\end{proof}

We can now establish \(\remvalue/n = \valuations[ \goodsordered{2n+1}, \dots, \goodsordered{m} ]/n\) as lower bound on  the valuations of items assigned after the first matching.
\begin{lemma}
	\label{lem:lower_bound_later_items}
	For each agent \(i \in \agents\) and her final bundle \(\alloc = \{\asgd{1}, \dots, \asgd{\alloclen}\}\), it holds \(\valuations[ \asgd{2}, \dots, \asgd{\alloclen} ] \ge \remvalue/n\).
\end{lemma}
\begin{proof}
	By \cref{lem:lower_bound_single_item} and definition of \(\goodsordered{\cdot}\), every item \(\asgd{t}\) is worth at least as much as each item \(\goodsordered{tn+k}\) with \(k \in \{1, \dots, n\}\) and, consequently, its valuation \(\valuations[\asgd{t}]\) is at least as high as the mean valuation \(\frac{1}{n} \valuations[ \goodsordered{tn+1}, \dots, \goodsordered{tn + n} ]\).
	Further, it holds \(\alloclen n + n \ge m \) since agent~\(i\) gets assigned \(\alloclen \ge \floor{\frac{m}{n}} \ge \frac{m}{n} - 1\) many items.
	Together, this yields\todo[info]{\(\mathbfit{i}\): I do not get the reason for the extra step in the original paper.}
	\begin{equation}
		\valuations[\asgd{2}, \dots, \asgd{\alloclen}]
		= \sum_{t=2}^{\alloclen \mathstrut} \valuations[\asgd{t}]
		\ge \sum_{t=2}^{\alloclen \mathstrut} \frac{1}{n} \valuations[ \goodsordered{tn+1}, \dots, \goodsordered{tn + n} ]
		\ge \frac{1}{n} \valuations[ \goodsordered{2n+1}, \dots, \goodsordered{m} ]
		= \frac{\remvalue}{n} \label{eq:lower_bound_later_items}.
	\end{equation}
\end{proof}

\begin{remark}
	In \cref{lem:lower_bound_later_items}, we assumed non-zero valuations for all items, hence the bundle lengths of \(\alloclen \ge \floor{\frac{m}{n}}\).
	Of course in an actual program, one would not assign items to agents who value them at zero.
	Nevertheless, \cref{lem:lower_bound_later_items} still holds inasmuch as additional zero valuations in \cref{eq:lower_bound_later_items} do not change the sum.
\end{remark}

This allows us to calculate an approximation factor for \SMatch{} by comparing its output with an optimal allocation \(\alloc*[][]\).
\begin{theorem}
	\label{th:smatch}
	\SMatch{} has an approximation factor of \(2 n\).
\end{theorem}
\begin{proof}
	\Cref{lem:lower_bound_later_items} can be plugged into the logarithmic \NSW:
	\begin{align}
		\log \NSW(\alloc[][])
		&= \frac{1}{\sum_{i \in \agents} \weight} \cdot \sum_{i \in \agents} \weight \log \valuations[\asgd{1}, \dots, \asgd{\alloclen}] \\
		&= \frac{1}{\sum_{i \in \agents} \weight} \cdot \sum_{i \in \agents} \weight \log \paren[\big]{ \valuations[\asgd{1}] + \valuations[\asgd{2}, \dots, \asgd{\alloclen}] } \\
		&\ge \frac{1}{\sum_{i \in \agents} \weight} \cdot \sum_{i \in \agents} \weight \log \paren[\big]{ \valuations[\asgd{1}] + \remvalue / n } \label{eq:smatch_maximising}
	\end{align}
	Notice that the first matching of \SMatch{} maximises the sum in \cref{eq:smatch_maximising}.
	Thus, assigning all agents \(i\) their respective most highly valued item \(\asgd*{1}\) in an optimal bundle~\(\alloc* = \{ \asgd*{1}, \dots, \asgd*{\alloclen*} \}\) yields the even lower bound
	\begin{equation}
		\label{eq:smatch_approx_factor_lower_bound}
		\log \NSW(\alloc[][])
		\ge \frac{1}{\sum_{i \in \agents} \weight} \cdot \sum_{i \in \agents} \weight \log \paren[\big]{ \valuations[\asgd*{1}] + \remvalue / n }.
	\end{equation}
	Recall the definition of \(\remvalue\) from \cref{eq:def_remvalue}.
	Consider a slightly modified variant:
	\begin{equation}
		%\remvalue = \smashoperator{\min_{\substack{\genericset \subset \goods \\ \abs{\genericset} \le 2n}}} \{ \valuations[\goods \setminus \genericset] \}
		%\textnormal{\quad or, alternatively,\quad}
		\remvalue = \valuations[\goods \setminus \genericset[i]]
		\textnormal{ with }
		\genericset[i] \coloneq \smashoperator{\argmin_{\substack{\genericset \subset \goods \\ \abs{\genericset} \le 2n}}} \{ \valuations[\goods \setminus \genericset] \}
	\end{equation}
	Moreover, consider the set \(\genericset*[i]\) of the (at most) \(2n\) most highly valued items in the optimal bundle \(\alloc*\), \ie
	\begin{equation}
		\genericset*[i]
		\coloneq \smashoperator{\argmin_{\substack{\genericset \subset \goods \\ \abs{\genericset} \le 2n}}} \{ \valuations[\alloc* \setminus \genericset] \}.
	\end{equation}
	It holds \(\valuations[\asgd*{1}] \ge \frac{1}{2 n} \valuations[\genericset*[i]]\) because of \(\valuations[\asgd*{1}] \ge \valuations[j]\) for all \(j \in \genericset*[i]\).
	Furthermore0, it holds \(\remvalue = \valuations[\goods \setminus \genericset[i]] \ge \valuations[\alloc* \setminus \genericset[i]] \ge \valuations[\alloc* \setminus \genericset*[i]]\).
	We can insert these two inequalities into \cref{eq:smatch_approx_factor_lower_bound} and prove the theorem thereby:
	\begin{align}
		\log \NSW(\alloc[][])
		&\ge \frac{1}{\sum_{i \in \agents} \weight} \cdot \sum_{i \in \agents} \weight \log \paren[\bigg]{ \frac{\valuations[\genericset*[i]]}{2 n} + \frac{\valuations[\alloc* \setminus \genericset*[i]]}{n} } \\
		&\ge \frac{1}{\sum_{i \in \agents} \weight} \cdot \sum_{i \in \agents} \weight \log \paren[\bigg]{ \frac{\valuations[\alloc*]}{2n} } \\
		&= \log \paren[\bigg]{\frac{\NSW(\alloc*[][])}{2 n}}
	\end{align}
	\vspace*{-5mm}
	\qedhere
\end{proof}

The analysis is asymptotically tight.
It is possible to design an instance for the asym\-metric \NSW{} such that \SMatch{} achieves an approximation ratio approaching \(2/n\).
It remains to be shown whether the symmetric \NSW{} is equally hard.~\cite[Section 6.3]{APNSWuSVþUM}

\begin{remark}
	\SMatch{} produces \emph{fair} allocations which are \emph{envy-free up to one item} (\abb{EF1}).
	An allocation \(\alloc[][]\) is \abb{EF1} if, for every pair \((i_1, i_2) \in \agents^2\) of agents, one needs to remove at most one item from the bundle \(\alloc[][i_2]\) of agent \(i_2\) so that agent \(i_1\) does not want to swap bundles.
	In other words, either it holds \(\valuations[\alloc[][i_1]][][i_1] \ge \valuations[\alloc[][i_2]][][i_2]\) or there is an item \(j \in \alloc[][i_2]\) such that \(\valuations[\alloc[][i_1]][][i_1] \ge \valuations[\alloc[][i_2] \setminus \{j\}][][i_2]\).~\cite[Section 5.2]{APNSWuSVþUM}
\end{remark}