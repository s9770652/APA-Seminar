\begin{abstract}
	Distributing goods fairly and efficiently between several recipients is of great interest in computer science, economics, politics, and many other areas.
	One common problem is the maximum Nash social welfare problem (\NSW).
	The \NSW{} asks to distribute unsharable and indivisible goods amongst recipients such that the geometric mean of the recipients' valuations for their assigned goods is maximised.
	In the asymmetric variant, the valuations are additionally multiplied by individual weights.

	\Citeauthor{APNSWuSVþUM}~\cite{APNSWuSVþUM} provide two new polynomial-time approximation algorithms for the \NSW.
	Contrary to previously known algorithms, the approximation guarantees depend only on the number of recipients but not on the number of goods even in the asymmetric variant.
	More specifically, the algorithms have approximation factors of \(\bigo(n)\) and \(\bigo(n \log n)\) for the asymmetric \NSW{} under additive and submodular valuation functions, respectively, where \(n\) is the number of recipients.

	In this seminar report on the paper by \citeauthor{APNSWuSVþUM}, we present both algorithms and analyse their efficiency.
	Furthermore, we prove that the \NSW{} under submodular valuation functions cannot be approximated by a factor better than \(\frac{\euler}{\euler - 1}\), even when not asymmetric.
\end{abstract}