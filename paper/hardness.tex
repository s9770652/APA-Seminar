\section{Hardness of Approximation}
\label{sec:hardness}

\Citeauthor{APNSWuSVþUM}~\cite[Sction 4]{APNSWuSVþUM} provide the following hardness result.
\begin{theorem}
	The submodular \NSW{} is not approximable by a factor better than \(\af\) in polynomial time unless \(\compP = \compNP\), even when the agents have equal weights and valuation functions.
\end{theorem}
\begin{proof}
	Consider the related maximum utilitarian social welfare problem\footnotemark.
	\begin{problem}
		\label{prob:usw}
		Given a set \(\goods\) of items and a set \(\agents\) of agents with submodular valuation functions \(\valuations \colon \powerset{\goods} \to \realposzero\) for all agents \(i \in \agents\), the symmetric submodular \emph{maximum utilitarian social welfare problem (\USW)} is to find an allocation \(\alloc*[][]\) which maximises the sum of valuations, that is,
		\begin{equation*}
			\alloc*[][] \overset{!}{=} \smashoperator{\argmax_{\alloc[][] \in \allallocs{\scriptstyle\agents\kern1pt}{\goods}}} \braces{ \USW(\alloc[][]) }
			\quad\text{with~}
			\USW(\alloc[][]) \coloneq \smashoperator{\sum_{i \in \agents}} \valuations[\alloc]
		\end{equation*}
		where \(\allallocs{\agents}{\goods}\) is the set of all possible allocations of the items in \(\goods\) amongst \(n\) agents.
	\end{problem}

	Note that this problem is identical to the symmetric submodular \NSW{} except for the sum in the objective function instead of a product.
	We are going to exploit this fact to calculate the Nash social welfare of instances for \USW.
	\citeauthor{inapprox_results_for_combi_auctions_with_submod_utility_funcs}~\cite{inapprox_results_for_combi_auctions_with_submod_utility_funcs} supply a polynomial-time reduction of \USW{} from the following problem:
	\begin{problem}
		Given a graph \(G = (V, E)\) and a constant \(\colouringconstant < 1\), the \emph{\(\colouringconstant\)-\Gap{} problem} is to decide whether, for any \(3\)-colouring of graph \(G\) which maximises the number of edges with differently coloured endpoints, the number of such edges is~\(\abs{E}\) (\emph{\Yes}) or \(\colouringconstant \abs{E}\) and below (\emph{\No}).
	\end{problem}

	\begin{proposition}
		There exists a constant \(\colouringconstant < 1\) such that the \(\colouringconstant\)-\Gap{} problem is \compNP\kern1.5pt-hard.
	\end{proposition}

	\medskip
	Reducing an instance of the \(c\)-\Gap{} problem yields an instance of the symmetric submodular \USW{} with identical valuation functions.
	Its properties are as follows:
	\begin{description}
		\item[\Yes]
		The utilitarian social welfare is \(n \USWconstant\) because every agent values her bundle at \(n\), whereby \(\USWconstant\) is a constant depending on the input graph.
		The Nash social welfare of the instance is \(\USWconstant\).

		\item[\No]
		The utilitarian social welfare is no more than \(\afinv \cdot n \USWconstant\) with some \(\USWsmallconstant > 0\).
		Applying the inequality of arithmetic and geometric means, \ie{}, \((x_1 + \dots + x_n)/n \ge \smash{\sqrt[n]{x_1 \dotsm x_n}}\) for all non-negative numbers \(x_1, \dots, x_n \in \realposzero\), reveals that the Nash social welfare of the instance is at most \(\afinv \cdot \USWconstant\).
	\end{description}
	Thereout follows that the submodular \NSW{}, even when symmetric and under identical valuation functions, cannot be approximated by a factor of \(\afconst\), so \(\af\) constitutes a lower bound on the approximation factor.
\end{proof}

For a constant number of agents, \citeauthor{APNSWuSVþUM}~\cite[Section 5.1]{APNSWuSVþUM} describe a family \((A_{\epsilon})_{\epsilon > 0}\) of algorithms for the asymmetric submodular \NSW{} where each algorithm \(A_\epsilon\) achieves an approximation factor of \(\af + \epsilon\).

\footnotetext{
	\Citeauthor{APNSWuSVþUM} (and many others) call \cref{prob:usw} the \enquote{Allocation problem}.
	We changed the name to match the naming scheme of \cref{prob:nsw} and to avoid confusion, as both problems are about finding allocations.
}