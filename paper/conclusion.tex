\section{Conclusion and Outlook}
\label{sec:conclusion}

In this seminar report, we presented two algorithms for the asymmetric Nash social welfare problem (\NSW), based on a paper by \citeauthor{APNSWuSVþUM}~\cite{APNSWuSVþUM}.
The asymmetric \NSW{} asks to find an allocation of unsharable and indivisible items amongst agents such that the weighted geometric mean of their valuations is maximised.
The novelty lies in both algorithms having an approximation factor dependant on the number \(n\) of agents but not on the number \(m\) of items.

The first algorithm, \SMatch, finds a \(2n\)-approximative allocation if the valuation functions are additive.
It does so by repeatedly matching agents with items within a weighted bipartite graph.
For the very first matching, the edge weights incorporate an estimation of the valuation of future items.

The second algorithm, \RepReMatch, finds a \(\frac{1}{2n(\log_2 n + 3)}\)-approximative allocation if the valuation functions are submodular.
In phase \phasei, a set of high-value items is determined through repeated matchings and then put away.
In phase \phaseii, agents are repeatedly matched with the remaining items.
In phase \phaseiii, the high-value items are finally assigned to the agents.

Lastly, it was shown that any polynomial-time algorithm for the submodular \NSW{} must have an approximation factor of at least \(\frac{\euler}{\euler - 1}\) unless \(\P = \NP\).
This holds even when the problem is symmetric and the valuation functions equal.

\medskip

The research on the \NSW{} gained pace in the last years.
Since the publication of the paper of \citeauthor{APNSWuSVþUM} at the start of 2020, quite a few new papers advancing the topic have been published.
\begin{itemize}[leftmargin=*, nosep]
	\item
	Rado valuations are a special class of submodular functions and stem from a generalisation of \OXS{} valuations.
	\Citeauthor{APNSWuSVþUM} emphatically mentioned the lack of research into the \OXS{} \NSW.
	\Citeauthor{approximating_nsw_under_rado_valuations}~\cite{approximating_nsw_under_rado_valuations} developed an algorithm for the Rado \NSW, which is independent from both the number of items and the number of agents.
	%
	For the symmetric Rado \NSW, their algorithm achieves an approximation factor of \(256 \euler^{3/\euler} \approx 772\).
	For the asymmetric Rado \NSW, the approximation factor is \(256 \gamma^3\) with \(\gamma \coloneq \max_{i \in \agents} \braces{\weight} / \min_{i \in \agents} \braces{\weight}\).
	Interestingly, the algorithm is \(16 \gamma\)-approximative in case of the asymmetric additive \NSW, so it outperforms \SMatch{} in many instances.
	Remarkably enough, the algorithm is divided into five phases, and the first phase serves to determine a set of high-value items.
%	Other than that, heavy use of linear programming is made.

	\item
	\citeauthor{a_constfactor_approx_algo_for_nsw_with_submod_valuations}~\cite{a_constfactor_approx_algo_for_nsw_with_submod_valuations} introduced a randomised, \(380\)-approximative algorithm for the symmetric submodular \NSW.
	Later, \citeauthor{approx_nsw_by_matching_and_local_search}~\cite{approx_nsw_by_matching_and_local_search} devised a family of deterministic algorithms any \(\epsilon > 0\).
	%, whose running times are polynomial in \(n\), \(m\), and \(1/\euler\).
	They are \((4+\epsilon)\)-approximative in the symmetric case and \(\euler(n \cdot \max_{i \in \agents} \braces{\weight} + 2 + \epsilon)\)-approximative in the asymmetric one.

	\item
	\XOS{} functions are a superclass of submodular functions, for which the approximation factor of \RepReMatch{} is not independent of the number of items anymore.
	\Citeauthor{sublin_approx_algo_for_nsw_with_xos_valuations}~\cite{sublin_approx_algo_for_nsw_with_xos_valuations} used both \RepReMatch{} and the discrete moving-knife method to get an \(\bigo(n^{53/54})\)-approximative algorithm for the symmetric \XOS{} \NSW.
	However, the algorithm uses demand and \XOS{} queries, whereas \RepReMatch{} needs only the weaker value queries.

	\item
	\CASC{} (also known as \SPLC) functions are a superclass of additive functions, which can model diminishing returns.
	\Citeauthor{fair_division_of_indiv_goods_for_a_class_of_concave_valuations}~\cite{fair_division_of_indiv_goods_for_a_class_of_concave_valuations} came up with an \(\euler^{1/\euler}\)-approximative algorithm for the symmetric \CASC{} \NSW.
	For \SMatch{}, only an approximation factor of \(2n\) for the symmetric \CASC{} \NSW{} could be shown by \citeauthor{APNSWuSVþUM}, although they suspect a better one.
\end{itemize}

Besides efficiency, fairness is also a property towards which algorithm can be designed.
Especially \RepReMatch{} with no fairness guarantees is unsatisfactory in that regard.
Research is made complex because of the myriad of notions of fairness.
A selection of new developments:
\begin{itemize}[leftmargin=*, nosep]
	\item
	\Citeauthor{approx_nsw_by_matching_and_local_search} guarantee \(\sfrac{1}{2}\)-EFX and an approximation factor of \(8 + \epsilon\).
	\todo[inline]{[24] improved efficiency}

	\item
	The algorithm of \citeauthor{fair_division_of_indiv_goods_for_a_class_of_concave_valuations} computes an allocation which is \(\sfrac{1}{2}\)-\EFone{} and approximately Pareto optimal.
\end{itemize}

\hrule

\begin{itemize}
%		\item
%		\cite{approximating_nsw_under_rado_valuations}:
%		Rado valuations = class of special submodular functions;
%		sym NSW = \(256\euler^{3/\euler} \approx 772\);
%		\(\weight \in [1, \gamma-1]\) -> asym NSW = \(256\gamma^{3}\), asym add NSW = \(16 \gamma\);
%		five phases where phase I computes a set of high-value items and phase II assigns the rest, other phases refinement (ie. recombining / release), heavy use of Linear programming;
%		gross substitute valuations: include prices, superclass of Rado

%		\item
%		\cite{a_constfactor_approx_algo_for_nsw_with_submod_valuations}:
%		randomised 380 for sym submod NSW;
%		five-phase approach, again with division into high-value items and others;
%		built upon \cite{approximating_nsw_under_rado_valuations}

%		\item
%		\cite{sublin_approx_algo_for_nsw_with_xos_valuations}:
%		sym XOS = superclass of submodular functions;
%		four-phase approach \(\bigo(n^{53/54})\) using reprematch and the discrete moving-knife method;
%		\(\frac{\euler}{\euler - 1}\) would require an exponential number of queries in value oracle model, therefore demand and XOS queries;
%		built upon \cite{approximating_nsw_under_rado_valuations}

%		\item
%		\cite{approx_nsw_by_matching_and_local_search}:
%		improves \cite{a_constfactor_approx_algo_for_nsw_with_submod_valuations};
%		det \(4 + \epsilon\) for sym submod NSW;
%		det \((\omega + 2 + \epsilon)\euler\) for asym submod NSW where \(\omega = n \cdot \max \weight\);
%		all polynomial in \(1/\epsilon\);

%		\item
%		\cite{fair_division_of_indiv_goods_for_a_class_of_concave_valuations}:
%		SPLC/CASC \(\euler^{1/\euler}\)-approx NSW 1/2-approx EF1, 1-approx PO; SMatch would be \(2n\)

		\item
		conclusion of \cite{min_envy_and_max_avg_nsw_in_the_alloc_of_indiv_goods}

		\item
		cf. \cite{min_envy_and_max_avg_nsw_in_the_alloc_of_indiv_goods}: do not only approximate efficiency but also fairness
		\begin{itemize}
%			\item
%			\cite{approx_nsw_by_matching_and_local_search}:
%			\(\sfrac{1}{2}\)-EFX and \(8 + \epsilon\)-approx for sym submod NSW, polynomial in \(1/\epsilon\)

			\item
			\cite{approx_nsw_by_matching_and_local_search}:
			\enquote{Prior work on EFX and related notions The existence of EFX allocations has not been
				settled despite significant efforts [14,17,43,44]. This problem is open for more than two agents with
				general monotone valuations (including submodular), and for more than three agents with additive
				valuations. This necessitated the study of its relaxations α-EFX for α ∈ (0, 1) and partial EFX
				allocations. For the notion of α-EFX, the best-known α is 0.618 for additive [1] and 0.5 for general
				monotone valuations (including submodular) [43].
				For the notion of partial EFX allocations, the existence is known for general monotone valuations
				if we do not allocate at most n − 2 items [9, 19, 40], albeit without any efficiency guarantees. For
				additive valuations, although n − 2 is still the best bound known, there exist partial EFX allocations
				with 2-approximation to the NSW problem [13].

				A well-studied weaker notion is envy-freeness up to one item (EF1), where no agent envies
				another agent after the removal of some item from the envied agent’s bundle. EF1 allocations are
				known to exist for general monotone valuations and can also be computed in polynomial-time [39].
				However, an EF1 allocation alone is not desirable because it might be highly inefficient in terms
				of any welfare objective. For additive valuations, the allocations maximizing NSW are EF1 [14].
				Although the NSW problem is APX-hard [36], there exists a pseduopolynomial time algorithm
				to find an allocation that is EF1 and 1.45-approximation to the NSW problem under additive
				valuations [8]. For capped-SPLC valuations, [16] shows the existence of an allocation that is 1/2-
				EF1 and 1.45-approximation to the NSW problem. The existence of an EF1 allocation with high
				NSW is open for submodular valuations.
				Subsequent to our work, [24] improves Theorem 1.2 to show the existence of an allocation T
				that is 1/2-EFX and NSW(T ) ≥ 2/3 NSW(S) for a given allocation S.}

			\item
			\cite{bobw_agents_with_entitlements}:
			Best of Both Worlds = combining randomisation and relaxation;
			asym add NSW;
			extendable to more general NSW like asym XOS NSW though ;
			eg. weighted proportional up to one good (WPROP1) := \(\valuations[\alloc \cup \{j\}] \ge \weight \valuations[\goods]\), weighted envy-free := \(\weight[j] \valuations[\alloc] \ge \weight \valuations[\alloc[][j]]\);
			invoke the image of agents eating their goods proportional speed to weights

			many notions of fairness actually exist
		\end{itemize}

		\item
		gap between hardness of add NSW and SMatch

		\item
		is approx factor for sym add NSW equally bad?

		\item
		RepReMatch does not extend to XOS or subadditive

		\item
		https://www.semanticscholar.org/reader/bef4514574543720bc45ea235df4c4556e97efe0

		\item
		\cite{fair_division_of_indiv_goods_for_a_class_of_concave_valuations}:
		\enquote{There is a considerable gap inthe lower bound (1.069) on the approximation factor that is hard to achieve (see Garg et al.(2019)) and the approximation factor of 1.445, even for the additive valuations.  It will beinteresting to close this gap.}
\end{itemize}