\section{Conclusion}
\label{sec:conclusion}

\subsection{Summary}
\label{subsec:conclusion:summary}

The asymmetric maximum Nash social welfare problem (\NSW) asks to find an allocation of unsharable and indivisible items amongst agents such that the weighted geometric mean of their valuations is maximised.
Based on a paper by \citeauthor{APNSWuSVþUM}~\cite{APNSWuSVþUM}, we presented two polynomial-time algorithms for the asymmetric \NSW.
The novelty of both algorithms lies in their approximation factors depending on the number \(n\) of agents but not on the number of~items.

The first algorithm, \SMatch, finds a \(2n\)-approximative allocation under additive valuation functions.
It does so by repeatedly matching agents with items within a weighted bipartite graph.
For the very first matching, the edge weights incorporate an estimation of the valuation of future items.
The output allocation is envy-free up to one item.

The second algorithm, \RepReMatch, finds a \(2n(\log_2 n + 3)\)-approximative allocation under submodular valuation functions.
In phase \phasei, a set of high-value items is determined through repeated matchings and then put away.
In phase \phaseii, agents are repeatedly matched with the remaining items.
In phase \phaseiii, the high-value items are finally assigned to the agents.

Additionally, it was shown that any polynomial-time algorithm for the submodular \NSW{} must have an approximation factor of at least \(\frac{\euler}{\euler - 1}\) unless \(\compP = \compNP\).
This holds even when the problem is symmetric and the valuation functions equal.

\subsection{Open Questions and Recent Work}
\label{subsec:conclusion:outlook}

\Citeauthor{APNSWuSVþUM}~\cite{APNSWuSVþUM} conjecture that \SMatch{} has a better approximation factor for the symmetric additive \NSW{}, though neither were they able to prove a better factor nor could they prove the tightness of their analysis.
Additionally, they forbore to prove the tightness of the analysis of \RepReMatch{}, and we suspect there to be quite some constants to be saved \Dash admittedly, the benefits of a more thorough analysis are questionable.

The research on the \NSW{} gained pace in the last years, and some publications are of especial interest in the context of this report:
\begin{itemize}
	\item
	\XOS{} functions encompass the submodular functions.
	Unfortunately, \RepReMatch{} does not guarantee an approximation factor independent of the number of items even for the symmetric \XOS{} \NSW~\cite[Section 6.2]{APNSWuSVþUM}.
	\Citeauthor{sublin_approx_algo_for_nsw_with_xos_valuations}~\cite{sublin_approx_algo_for_nsw_with_xos_valuations} used both a modified \RepReMatch{} and the discrete moving-knife method to get an \(\bigomicron{n^{53/54}}\)-approximation algorithm.
	However, the algorithm uses demand and \XOS{} queries, whereas \RepReMatch{} needs only the weaker value queries.

	\item
	For the submodular \NSW, \citeauthor{approx_nsw_by_matching_and_local_search}~\cite{approx_nsw_by_matching_and_local_search} devised a family \((A_\epsilon)_{\epsilon > 0}\) of algorithms which are \((4+\epsilon)\)-approximative in the symmetric case and \(\euler(n \cdot \max_{i \in \agents} \,\braces{\weight} + 2 + \epsilon)\)-approximative in the asymmetric one.

	\item
	Rado valuations are a special class of submodular functions and stem from a generalisation of \OXS{} valuations, in regard to which \citeauthor{APNSWuSVþUM} explicitly mentioned a lack of research.
	\Citeauthor{approximating_nsw_under_rado_valuations}~\cite{approximating_nsw_under_rado_valuations} developed a five-phase algorithm, which shares ideas with \RepReMatch.
	It achieves an approximation factor of \(256 \euler^{3/\euler} \approx 772\) in case of the symmetric Rado \NSW{} and of \(256 \gamma^3\) with \(\gamma \coloneq \max_{i \in \agents} \,\braces{\weight} / \min_{i \in \agents} \,\braces{\weight}\) in the asymmetric case.
	Interestingly, the factor is \(16 \gamma\) in case of the asymmetric additive \NSW, so the algorithm outperforms both \SMatch{} and \RepReMatch{} in many instances.
\end{itemize}
Besides efficiency, fairness is also a property towards which algorithms can be designed, although \SMatch{} and \RepReMatch{} advance nothing in that regard.
Apart from devising more algorithms for more valuation functions or with features inspired by reality, it could also be rewarding to look at basic cases.
For example, the gap between the hardness of \(1.069\)~\cite{satiation_in_fisher_markets_and_approx_of_nsw} and the best currently achievable approximation factor of \(1.45\)~\cite{finding_fair_and_efficient_allocs} of the symmetric additive \NSW{} has yet to be closed.