\section{RepReMatch}

\subsection{Naïve Approach}
\begin{frame}{Naïve Approach}
	\tikzset{
		agent/.style = {draw, circle, font=\small, inner sep=1pt},
		item/.style = {draw, rectangle, font=\small},
		dots/.style = {font=\small},
		weight/.style = {font=\footnotesize},
		edgeopt/.style = {},
		edgealg/.style = {},
		edgegry/.style = {},
		node distance=12mm and 32mm,
		on grid
	}
	\def\allocationexample{
		\centering
		\begin{tikzpicture}
			% items
			\node[item] (i1)                       {\(\goods_{1}\)};
			\node[item] (i2)  [below=of i1]        {\(\goods_{2}\)};
			\node[item] (i3)  [below=of i2]        {\(\goods_{3}\)};
			\node[item] (im)  [below=of i3]        {\(\goods_{m}\)};
			\node[item] (im1) [below=of im]        {\(\goods_{m+1}\)};
			\node[dots] (id)  at ($(i3)!0.5!(im)$) {\(\vdots\)};
			% agents
			\node[agent] (a1) [left=of i1]  {\(\agents_1\)};
			\node[agent] (a2) [left=of im1] {\(\agents_2\)};
			% valuations
			\draw[edgealg] (a1) to node[weight, pos=.42,  above]      {\(m + \epsilon\)} (i1);
			\draw          (a1) to node[weight, pos=.42,  above]      {\(1\)}            (i2);
			\draw          (a1) to node[weight, pos=.39,  above]      {\(1\)}            (i3);
			\draw          (a1) to node[weight, pos=.32,  above]      {\(1\)}            (im);
			\draw[edgeopt] (a1) to node[weight, pos=.255, above]      {\(1\)}            (im1);

			\draw[edgeopt] (a2) to node[weight, pos=.26,  above left] {\(m\)}            (i1);
			\draw[edgegry] (a2) to node[weight, pos=.30,  above]      {\(0\)}            (i2);
			\draw[edgegry] (a2) to node[weight, pos=.37,  above]      {\(0\)}            (i3);
			\draw[edgegry] (a2) to node[weight, pos=.42,  above]      {\(0\)}            (im);
			\draw[edgealg] (a2) to node[weight, pos=.48,  above]      {\(1\)}            (im1);
		\end{tikzpicture}
	}
	\begin{center}
		\allocationexample
	\end{center}
	\begin{block}{}
		What are the low-value items?\phantom{g}  % 'g' has an ascender.
	\end{block}
\end{frame}

\begin{frame}{Looking into the Future}
	\begin{itemize}
		\item
		sort items by valuation in descending order
		\begin{itemize}
			\item
			low-value items on the left
		\end{itemize}
	\end{itemize}
	\begin{center}
		\includegraphics[width=10cm, height=2cm, page=21]{example-image-duck}
	\end{center}
	\begin{itemize}
		\item
		use their valuations for edge weights in early matchings
	\end{itemize}
	\begin{exampleblock}{}
		A \(2n\)-approximation is possible \dots{} using \SMatch.
	\end{exampleblock}
	\begin{alertblock}{}
		This only works for additive valuation functions.
	\end{alertblock}
\end{frame}

\begin{frame}{Changing the Past}
	Under submodular valuation, the set of lowest valuation is approximable only by \(\bigomega\paren[\big]{ \sqrt{m/\ln m} }\).
	\includegraphics[height=3ex, page=66]{example-image-duck}

	We can change the past in three phases:
	\begin{description}
		\item[Phase \phasei]
		Assign enough high-value items temporarily.

		\item[Phase \phaseii]
		Assign the remaining items definitely.

		\item[Phase \phaseiii]
		Re-assign the items of phase \phasei{} definitely.
	\end{description}
	\begin{center}
		\includegraphics[height=3cm, page=43]{example-image-duck}
	\end{center}
	\begin{exampleblock}{}
		A \(2n (\log_2 n + 3)\)-approximation is possible!
	\end{exampleblock}
\end{frame}





\subsection{The Algorithm}
\begin{frame}{The Algorithm}
	Phase \phasei
	\begin{enumerate}
		\item
		repeat \(\ceil{\log_2 n} + 1\) times or until \(\goods = \emptyset\)
		\begin{enumerate}
			\item
			create bipartite graph \(\bipartitegraph = (\agents, \goods, E)\) with edge weights \(\edgeweight{i, j} = \weight \log \valuations[ j ]\)

			\item
			compute maximum weight matching \(\matching\)

			\item
			update bundles \(\alloc[\phasei]\) according to matching \(\matching\) and remove assigned items
		\end{enumerate}
		\seti
	\end{enumerate}
	Phase \phaseii
	\begin{enumerate}
		\conti
		\item
		repeat until \(\goods = \emptyset\)
		\begin{enumerate}
			\item
			create bipartite graph \(\bipartitegraph = (\agents, \goods, E)\) with edge weights \(\edgeweight{i, j} = \weight \log \paren[\big]{ \valuations[\alloc[\phaseii] \cup \{ j \} ] }\)

			\item
			compute maximum weight matching \(\matching\)

			\item
			update bundles \(\alloc[\phaseii]\) according to matching \(\matching\) and remove assigned items
		\end{enumerate}
		\seti
	\end{enumerate}
	Phase \phaseiii
	\begin{enumerate}
		\conti
		\item
		create bipartite graph \(\bipartitegraph = (\agents, \bigcup_{i \in \agents} \alloc[\phasei], E)\) with edge weights \(\edgeweight{i, j} = \weight \log \paren[\big]{ \valuations[\alloc[\phaseii] \cup \{ j \} ] }\)

		\item
		compute maximum weight matching \(\matching\)

		\item
		create bundles \(\alloc[\phaseiii]\) according to matching \(\matching\) and previous bundles \(\alloc[\phaseii]\)
	\end{enumerate}
\end{frame}





\subsection{Analysing Phase \texorpdfstring{\phaseii}{II}}
\begin{frame}{Analysing Phase \phaseii}
	\adjustforminipage
	\begin{minipage}{0.55\textwidth}
		\begin{definition}[9]
			The set \(\lostset{r}\) of \emph{lost items} is the set of all optimal items \(j \in \alloc*\) assigned to other agents \(i' \neq i\) in round \(r\).
		\end{definition}
		\begin{definition}[10]
			The set of \emph{optimal and attainable items} is defined as
			\begin{equation*}
				\attopt{r} \coloneq \begin{cases*}
					\alloc* \setminus \bigcup_{i' \in \agents} \alloc[\phasei][i'] & in round \(r = 0\), \\
					\attopt{0} \setminus \lostset{1} & in round \(r = 1\), \\
					\attopt{r-1} \setminus ( \lostset{r} \cup \{\asgd{r-1}\} ) & in round \(r = 2, \dots, \alloclen[\phaseii]\).
				\end{cases*}
			\end{equation*}
		\end{definition}

		\medskip

		\(\Rightarrow\) What is the valuation of the remaining items?
	\end{minipage}
	\beamerimage at (3.7cm, 0cm) {\includegraphics[height=3.75cm, page=25]{example-image-duck}};
\end{frame}

\begin{frame}{Analysing Phase \phaseii \Dash Valuation of Unassigned Items (1/4)}
	\adjustfortopblock
	\begin{lemma}[11]
		\begin{equation*}
			\valuations[ \attopt{r} \given \asgd{1}, \dots, \asgd{r-1} ] \ge \valuations[\attopt{1}] - \smashoperator{\sum_{r'=1}^{r-1}} \abs{\lostset{r'+1}} \cdot \valuations[ \asgd{r'} \given \asgd{1}, \dots, \asgd{r'-1} ] - \valuations[\asgd{1}, \dots, \asgd{r-1}]  \tag*{\(\forall r \ge 2\)}
		\end{equation*}
		\begin{itemize}
			\item
			\(\alloc[\phaseii] = \{ \asgd{1}, \dots, \asgd{\alloclen[\phaseii]} \}\)
		\end{itemize}
	\end{lemma}
	\begin{proof}
		\begin{itemize}
			\item
			definition of marginal valuation
		\end{itemize}
		\begin{align*}
			\valuations[ \attopt{r} \given \asgd{1}, \dots, \asgd{r-1} ]
			= \valuations[ \attopt{r} \cup \braces{ \asgd{1}, \dots, \asgd{r-1} } ] - \valuations[ \asgd{1}, \dots, \asgd{r-1} ]
		\end{align*}
		\(\Rightarrow\) We need a lower bound on \(\valuations[ \attopt{r} \cup \braces{ \asgd{1}, \dots, \asgd{r-1} } ]\)\mperiod[!]
		\renewcommand{\qedsymbol}{}
	\end{proof}
\end{frame}

\begin{frame}{Analysing Phase \phaseii \Dash Valuation of Unassigned Items (2/4)}
	\adjustfortopblock
	\begin{lemma}[11]
		\begin{equation*}
			\valuations[ \attopt{r} \given \asgd{1}, \dots, \asgd{r-1} ] \ge \valuations[\attopt{1}] - \smashoperator{\sum_{r'=1}^{r-1}} \abs{\lostset{r'+1}} \cdot \valuations[ \asgd{r'} \given \asgd{1}, \dots, \asgd{r'-1} ] - \valuations[\asgd{1}, \dots, \asgd{r-1}]  \tag*{\(\forall r \ge 2\)}
		\end{equation*}
	\end{lemma}
	\begin{proof}
		\begin{itemize}
			\item
			\(\attopt{r} = \attopt{r-1} \setminus ( \lostset{r} \cup \{\asgd{r-1}\} )\) \(\implies\) \(\paren{ \attopt{r} \cup \braces{\asgd{r-1}} } \supset \paren{ \attopt{r-1} \setminus \lostset{r} }\)
			\begin{itemize}
				\item
				item \(\asgd{r-1}\) perhaps not element of \(\attopt{r-1}\)
			\end{itemize}

			\item
			diminishing returns \(\implies\) \(\valuations[ \genericset[1] \given \genericset[2] \cup \genericset[3] ] \le \valuations[ \genericset[1] \given \genericset[2] ] \)
		\end{itemize}
		\vspace{-1.25ex}
		\begin{align*}
			\valuations[\attopt{r} \cup \braces{ \asgd{1}, \dots, \asgd{r-1} }]
			&\ge \valuations[\attopt{r-1} \setminus \lostset{r} \cup \braces{ \asgd{1}, \dots, \asgd{r-2} }] \\
			&= \valuations[\attopt{r-1} \cup \braces{ \asgd{1}, \dots, \asgd{r-2} }] - \valuations[ \lostset{r} \given \attopt{r-1} \setminus \lostset{r} \cup \braces{ \asgd{1}, \dots, \asgd{r-2} } ] \\
			&\ge \valuations[\attopt{r-1} \cup \braces{ \asgd{1}, \dots, \asgd{r-2} }] - \valuations[ \lostset{r} \given \asgd{1}, \dots, \asgd{r-2} ]
		\end{align*}
		\renewcommand{\qedsymbol}{}
		\vspace*{-4.5ex}
	\end{proof}
\end{frame}

\begin{frame}{Analysing Phase \phaseii \Dash Valuation of Unassigned Items (3/4)}
	\adjustfortopblock
	\begin{lemma}[11]
		\begin{equation*}
			\valuations[ \attopt{r} \given \asgd{1}, \dots, \asgd{r-1} ] \ge \valuations[\attopt{1}] - \smashoperator{\sum_{r'=1}^{r-1}} \abs{\lostset{r'+1}} \cdot \valuations[ \asgd{r'} \given \asgd{1}, \dots, \asgd{r'-1} ] - \valuations[\asgd{1}, \dots, \asgd{r-1}]  \tag*{\(\forall r \ge 2\)}
		\end{equation*}
	\end{lemma}
	\begin{proof}
		\begin{itemize}
			\item
			apply inequality recursively
		\end{itemize}
		\begin{align*}
			\valuations[\attopt{r} \cup \braces{ \asgd{1}, \dots, \asgd{r-1} }]
			&\ge \valuations[\attopt{r-1} \cup \braces{ \asgd{1}, \dots, \asgd{r-2} }] - \valuations[ \lostset{r} \given \asgd{1}, \dots, \asgd{r-2} ] \\
			&\ge \valuations[\attopt{1}] - \smashoperator{\sum_{r'=1}^{r-1}} \valuations[\lostset{r'+1} \given \asgd{1}, \dots, \asgd{r'-2} ]
		\end{align*}
		\(\Rightarrow\) We need an upper bound on \(\valuations[ \lostset{r'+1} \given \asgd{1}, \dots, \asgd{r'-2} ]\)\mperiod[!]
		\renewcommand{\qedsymbol}{}
	\end{proof}
\end{frame}

\begin{frame}{Analysing Phase \phaseii \Dash Valuation of Unassigned Items (4/4)}
	\adjustfortopblock
	\begin{lemma}[11]
		\begin{equation*}
			\valuations[ \attopt{r} \given \asgd{1}, \dots, \asgd{r-1} ] \ge \valuations[\attopt{1}] - \smashoperator{\sum_{r'=1}^{r-1}} \abs{\lostset{r'+1}} \cdot \valuations[ \asgd{r'} \given \asgd{1}, \dots, \asgd{r'-1} ] - \valuations[\asgd{1}, \dots, \asgd{r-1}]  \tag*{\(\forall r \ge 2\)}
		\end{equation*}
	\end{lemma}
	\begin{proof}
		\begin{itemize}
			\item
			diminishing returns \(\implies\) \(\valuations[\genericset] \le \sum_{\genericitem \in \genericset} \valuations[ \hairspace \genericitem ]\) for all \(\genericset\)

			\item
			item \(\asgd{r'}\) assigned before any item \(\genericitem \in \lostset{r'+1}\) \(\implies\) \(\valuations[\asgd{r'} \given \asgd{1}, \dots, \asgd{r'-1}] \ge \valuations[\hairspace \genericitem \given \asgd{1}, \dots, \asgd{r'-1}]\)
		\end{itemize}
		\begin{align*}
			\valuations[ \lostset{r'+1} \given \asgd{1}, \dots, \asgd{r'-2} ]
			&\le \smashoperator{ \sum_{\genericitem \in \lostset{r'+1}} } \valuations[ \hairspace \genericitem \given \asgd{1}, \dots, \asgd{r'-2} ] \\
			&\le \abs{\lostset{r'+1}} \cdot \valuations[\asgd{r'} \given \asgd{1}, \dots, \asgd{r'-1}] \tag*{\qedsymbol}
		\end{align*}
		\vspace*{-4.5ex}
		\renewcommand{\qedsymbol}{}
	\end{proof}
\end{frame}

\begin{frame}{Analysing Phase \phaseii \Dash Valuation of Assigned Items}
	\adjustfortopblock
	\begin{lemma}[13]
		\begin{equation*}
			\valuations[\asgd{1}, \dots, \asgd{\alloclen[\phaseii]}] \ge \valuations[\attopt{1}] / n \vphantom{\sum_{r'=1}^{r-1}}
		\end{equation*}
	\end{lemma}
	\begin{proof}
		Left as exercise to the listeners.

		\smallskip

		Hint: \(\abs{\lostset{r}} \le n - 1\)
	\end{proof}
\end{frame}





\subsection{Analysing Phases \texorpdfstring{\phasei{} \& \phaseiii}{I \& III}}
\begin{frame}{Analysing Phases \phasei{} \& \phaseiii}
	\adjustfortopblock
	\begin{block}{Reminder\vphantom{(15)}}
		\begin{description}
			\item[Phase \phasei]
			Temporary assignments.
			Valuations of single items as edge weights.

			\item[Phase \phaseiii]
			Items of phase \phasei{} released.
			Valuations of items and bundles from phase \phaseii{} as edge weights.
		\end{description}
	\end{block}

	Phase \phasei{} reserves \enquote{high-value} items.
	But what qualifies as \enquote{high-value}?

	\begin{definition}[14]
		Let \(\alloc* = \{ \asgd*{1}, \asgd*{2}, \dots\}\) be an optimal bundle.
		An item \(\genericitem \in \goods\) is \emph{outstanding} if \(\valuations[ \hairspace \genericitem] \ge \valuations[\asgd*{1}]\).
	\end{definition}

	\medskip
	\(\Rightarrow\) Are enough outstanding items reserved?
\end{frame}





\begin{frame}{Analysing Phases \phasei{} \& \phaseiii \Dash todo}
	\adjustfortopblock
	\begin{lemma}[15]
		Each agent can be matched with an outstanding item in phase \phaseiii.
	\end{lemma}
	\begin{proof}[Sketch Proof]
		\begin{columns}[T]
			\begin{column}{0.6\textwidth}
				\begin{itemize}
					\item
					number of agents not matched with an outstanding item in phase \phaseiii{} halved with each round of phase \phasei
					\begin{itemize}
						\item
						induction on number of rounds in phase \phasei
					\end{itemize}

					\item
					\(\ceil{\log_2 n} + 1\) rounds in phase \phasei{} are enough
				\end{itemize}
				Base Case: In round 1 of phase \phasei, either
				\begin{itemize}
					\item
					\(\ge n/2\) many agents matched with an outstanding item

					\item
					\(< n/2\) many agents matched with an outstanding item
					\begin{itemize}
						\item
						\(> n/2\) many items \(\asgd*{1}\) assigned to someone else

						\item
						\(> n/2\) many agents matched upon release in phase \phaseiii
					\end{itemize}
				\end{itemize}
			\end{column}
			\begin{column}{0.375\textwidth}
				\includegraphics[width=5cm, page=19]{example-image-duck}
			\end{column}
		\end{columns}
		\vspace{-2ex}
	\end{proof}
\end{frame}





\begin{frame}{The Approximation Factor}
	\begin{theorem}[16]
		The approximation factor is \(2n ( \log_2 n + 3 ) \).
	\end{theorem}
\end{frame}