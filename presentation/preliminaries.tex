\section{Preliminaries}

\subsection{Allocations}
\begin{frame}{Allocations}
	Setting:
	\begin{itemize}
		\item
		recipients: set \(\agents\) of \(n\) agents

		\item
		goods: set \(\goods\) of \(m\) items
%		\begin{itemize}
%			\item
%			unsharable
%
%			\item
%			indivisible
%		\end{itemize}
	\end{itemize}
%	\beamerimage at ( 8cm, 1.4cm) {\resizebox{!}{2cm}{\input{img/nosatellite.pdf_tex}}};
%	\beamerimage at (11cm, 1.4cm) {\resizebox{!}{2cm}{\input{img/nowater.pdf_tex}}};
%	\vspace{-3ex}
	\begin{definition}
		An \emph{allocation} is a tuple
		\(\alloc[][] = (\alloc)_{i \in \agents}\)
		of bundles \(\alloc \subset \goods\) such that each item is element of precisely one bundle.

		Item \(j\) is \emph{assigned} to agent \(i\) if \(j \in \alloc\).
	\end{definition}

	But how to measure its efficiency and fairness?
\end{frame}





\subsection{Valuation Functions}
\begin{frame}{Valuation Functions}{}
	Requirements:
	\begin{itemize}
		\item
		monotonically non-decreasing: \(\valuations[\genericset[1]] \le \valuations[\genericset[2]]\) if \(\genericset[1] \subset \genericset[2]\)

		\item
		normalised: \(\valuations[\emptyset] = 0\)
	\end{itemize}

	\medskip

	Types:
	\begin{itemize}
		\item
		additive: \(\valuations[\genericset] \coloneq \sum_{\genericitem \in \genericset} \valuations[ \hairspace \genericitem ]\)

		\item
		submodular: \(\valuations[\genericset[1] \given \genericset[2]] \coloneq \valuations[\genericset[1] \cup \genericset[2]] - \valuations[\genericset[2]]\)
		\begin{itemize}
			\item
			diminishing returns
		\end{itemize}
	\end{itemize}

	\begin{center}
		\resizebox{!}{2.5cm}{\input{img/additive.pdf_tex}}
		\hfil
		\resizebox{!}{2.5cm}{\input{img/submodular.pdf_tex}}
		\hfil
		\resizebox{!}{2.5cm}{\input{img/diminishingreturns.pdf_tex}}
	\end{center}
\end{frame}





\subsection{Maximum Nash Social Welfare Problem}
\begin{frame}{Asymmetric Maximum Nash Social Welfare Problem}
	\adjustfortopblock
	\begin{problem}
		\begin{equation*}
			\alloc*[][] \overset{!}{=} \smashoperator{\argmax_{\alloc[][] \in \allallocs{\scriptstyle\agents\kern1pt}{\goods}}} \braces{ \NSW(\alloc[][]) }
			\quad\text{with~}
			\NSW(\alloc[][]) \coloneq \paren[\Big]{ \smashoperator{\prod_{i \in \agents}} \valuations[\alloc]^{\,\textstyle\weight} }{}^{\textstyle 1 / \sum_{i \in \agents} \weight}
		\end{equation*}
		\begin{itemize}
			\item
			\(\allallocs{\agents\kern1pt}{\goods}\): all possible allocations

			\item
			\(\weight\): agent weight
		\end{itemize}
	\end{problem}
	The NSW strikes a middle ground between efficiency and fairness!
	\begin{alertblock}{}
		Is there a polynomial-time algorithm with an approximation factor \dots
		\begin{itemize}
			\item
			\dots{} dependent on \(n\)?

			\item
			\dots{} independent from \(m\)?
		\end{itemize}
		\def\svgwidth{3cm}
		\beamerimage at (13.5cm, 1.1cm) {\input{img/nvsm.pdf_tex}};
		\vspace{-0.75ex}
	\end{alertblock}
\end{frame}